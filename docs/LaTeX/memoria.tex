\documentclass[a4paper,11pt,oneside]{memoir}

% Castellano
\usepackage[spanish,es-tabla]{babel}
\selectlanguage{spanish}
\usepackage[utf8]{inputenc}
\usepackage[T1]{fontenc}
\usepackage{placeins}
\usepackage{enumitem}

\RequirePackage{xtab}
\RequirePackage{booktabs}
\RequirePackage[table]{xcolor}
\RequirePackage{multirow}

% Extra
\usepackage{float}
\usepackage{longtable}
\usepackage{amsfonts}
\usepackage{titlesec}

% Subsubsubsection

\titleformat{\paragraph}
{\normalfont\normalsize\bfseries}{\theparagraph}{1em}{}
\titlespacing*{\paragraph}
{0pt}{3.25ex plus 1ex minus .2ex}{1.5ex plus .2ex}

% Licencia
\usepackage[framemethod=tikz]{mdframed}
\definecolor{cccolor}{rgb}{.67,.7,.67}

% Links
\usepackage[colorlinks]{hyperref}
\hypersetup{
	colorlinks,
	linkcolor={black},
	citecolor={blue!50!black},
	urlcolor={blue!80!black}
}

% Ecuaciones
\usepackage{amsmath}

% Rutas de fichero / paquete
\newcommand{\ruta}[1]{{\sffamily #1}}

% Párrafos
\nonzeroparskip

% Formatos de enumeración (por niveles)
\setlist[itemize,1]{label=$\bullet$}
\setlist[itemize,2]{label=$\diamond$}
\setlist[itemize,3]{label=$\checkmark$}
\setlist[itemize,4]{label=$\times$}

% Imagenes
\usepackage{graphicx}
\newcommand{\imagen}[3]{
	\begin{figure}[!h]
		\centering
		\includegraphics[width=#3\textwidth]{#1}
		\caption{#2}\label{fig:#1}
	\end{figure}
	\FloatBarrier
}

\newcommand{\imagenflotante}[2]{
	\begin{figure}%[!h]
		\centering
		\includegraphics[width=0.9\textwidth]{#1}
		\caption{#2}\label{fig:#1}
	\end{figure}
}



% El comando \figura nos permite insertar figuras comodamente, y utilizando
% siempre el mismo formato. Los parametros son:
% 1 -> Porcentaje del ancho de página que ocupará la figura (de 0 a 1)
% 2 --> Fichero de la imagen
% 3 --> Texto a pie de imagen
% 4 --> Etiqueta (label) para referencias
% 5 --> Opciones que queramos pasarle al \includegraphics
% 6 --> Opciones de posicionamiento a pasarle a \begin{figure}
\newcommand{\figuraConPosicion}[6]{%
  \setlength{\anchoFloat}{#1\textwidth}%
  \addtolength{\anchoFloat}{-4\fboxsep}%
  \setlength{\anchoFigura}{\anchoFloat}%
  \begin{figure}[#6]
    \begin{center}%
      \Ovalbox{%
        \begin{minipage}{\anchoFloat}%
          \begin{center}%
            \includegraphics[width=\anchoFigura,#5]{#2}%
            \caption{#3}%
            \label{#4}%
          \end{center}%
        \end{minipage}
      }%
    \end{center}%
  \end{figure}%
}

%
% Comando para incluir imágenes en formato apaisado (sin marco).
\newcommand{\figuraApaisadaSinMarco}[5]{%
  \begin{figure}%
    \begin{center}%
    \includegraphics[angle=90,height=#1\textheight,#5]{#2}%
    \caption{#3}%
    \label{#4}%
    \end{center}%
  \end{figure}%
}
% Para las tablas
\newcommand{\otoprule}{\midrule [\heavyrulewidth]}
%
% Nuevo comando para tablas pequeñas (menos de una página).
\newcommand{\tablaSmall}[5]{%
 \begin{table}
  \begin{center}
   \rowcolors {2}{gray!35}{}
   \begin{tabular}{#2}
    \toprule
    #4
    \otoprule
    #5
    \bottomrule
   \end{tabular}
   \caption{#1}
   \label{tabla:#3}
  \end{center}
 \end{table}
}

%
% Nuevo comando para tablas pequeñas (menos de una página).
\newcommand{\tablaSmallSinColores}[5]{%
 \begin{table}[H]
  \begin{center}
   \begin{tabular}{#2}
    \toprule
    #4
    \otoprule
    #5
    \bottomrule
   \end{tabular}
   \caption{#1}
   \label{tabla:#3}
  \end{center}
 \end{table}
}

\newcommand{\tablaApaisadaSmall}[5]{%
\begin{landscape}
  \begin{table}
   \begin{center}
    \rowcolors {2}{gray!35}{}
    \begin{tabular}{#2}
     \toprule
     #4
     \otoprule
     #5
     \bottomrule
    \end{tabular}
    \caption{#1}
    \label{tabla:#3}
   \end{center}
  \end{table}
\end{landscape}
}

%
% Nuevo comando para tablas grandes con cabecera y filas alternas coloreadas en gris.
\newcommand{\tabla}[6]{%
  \begin{center}
    \tablefirsthead{
      \toprule
      #5
      \otoprule
    }
    \tablehead{
      \multicolumn{#3}{l}{\small\sl continúa desde la página anterior}\\
      \toprule
      #5
      \otoprule
    }
    \tabletail{
      \hline
      \multicolumn{#3}{r}{\small\sl continúa en la página siguiente}\\
    }
    \tablelasttail{
      \hline
    }
    \bottomcaption{#1}
    \rowcolors {2}{gray!35}{}
    \begin{xtabular}{#2}
      #6
      \bottomrule
    \end{xtabular}
    \label{tabla:#4}
  \end{center}
}

%
% Nuevo comando para tablas grandes con cabecera.
\newcommand{\tablaSinColores}[6]{%
  \begin{center}
    \tablefirsthead{
      \toprule
      #5
      \otoprule
    }
    \tablehead{
      \multicolumn{#3}{l}{\small\sl continúa desde la página anterior}\\
      \toprule
      #5
      \otoprule
    }
    \tabletail{
      \hline
      \multicolumn{#3}{r}{\small\sl continúa en la página siguiente}\\
    }
    \tablelasttail{
      \hline
    }
    \bottomcaption{#1}
    \begin{xtabular}{#2}
      #6
      \bottomrule
    \end{xtabular}
    \label{tabla:#4}
  \end{center}
}

%
% Nuevo comando para tablas grandes sin cabecera.
\newcommand{\tablaSinCabecera}[5]{%
  \begin{center}
    \tablefirsthead{
      \toprule
    }
    \tablehead{
      \multicolumn{#3}{l}{\small\sl continúa desde la página anterior}\\
      \hline
    }
    \tabletail{
      \hline
      \multicolumn{#3}{r}{\small\sl continúa en la página siguiente}\\
    }
    \tablelasttail{
      \hline
    }
    \bottomcaption{#1}
  \begin{xtabular}{#2}
    #5
   \bottomrule
  \end{xtabular}
  \label{tabla:#4}
  \end{center}
}



\definecolor{cgoLight}{HTML}{EEEEEE}
\definecolor{cgoExtralight}{HTML}{FFFFFF}

%
% Nuevo comando para tablas grandes sin cabecera.
\newcommand{\tablaSinCabeceraConBandas}[5]{%
  \begin{center}
    \tablefirsthead{
      \toprule
    }
    \tablehead{
      \multicolumn{#3}{l}{\small\sl continúa desde la página anterior}\\
      \hline
    }
    \tabletail{
      \hline
      \multicolumn{#3}{r}{\small\sl continúa en la página siguiente}\\
    }
    \tablelasttail{
      \hline
    }
    \bottomcaption{#1}
    \rowcolors[]{1}{cgoExtralight}{cgoLight}

  \begin{xtabular}{#2}
    #5
   \bottomrule
  \end{xtabular}
  \label{tabla:#4}
  \end{center}
}

\graphicspath{ {../sphinx/source/_static/images/} }

% Capítulos
\chapterstyle{bianchi}
\newcommand{\capitulo}[2]{
	\setcounter{chapter}{#1}
	\setcounter{section}{0}
	\chapter*{#2}
	\addcontentsline{toc}{chapter}{#2}
	\markboth{#2}{#2}
}

% Apéndices
\renewcommand{\appendixname}{Apéndice}
\renewcommand*\cftappendixname{\appendixname}

\newcommand{\apendice}[1]{
	%\renewcommand{\thechapter}{A}
	\chapter{#1}
}

\renewcommand*\cftappendixname{\appendixname\ }

% Formato de portada
\makeatletter
\usepackage{xcolor}
\newcommand{\tutor}[1]{\def\@tutor{#1}}
\newcommand{\course}[1]{\def\@course{#1}}
\definecolor{cpardoBox}{HTML}{E6E6FF}
\def\maketitle{
  \null
  \thispagestyle{empty}
  % Cabecera ----------------
\noindent\includegraphics[width=\textwidth]{cabecera}\vspace{1cm}%
  \vfill
  % Título proyecto y escudo informática ----------------
  \colorbox{cpardoBox}{%
    \begin{minipage}{.8\textwidth}
      \vspace{.5cm}\Large
      \begin{center}
      \textbf{TFG del Grado en Ingeniería Informática}\vspace{.6cm}\\
      \textbf{\LARGE\@title{}}
      \end{center}
      \vspace{.2cm}
    \end{minipage}

  }%
  \hfill\begin{minipage}{.20\textwidth}
    \includegraphics[width=\textwidth]{escudoInfor}
  \end{minipage}
  \vfill
  % Datos de alumno, curso y tutores ------------------
  \begin{center}%
  {%
    \noindent\LARGE
    Presentado por \@author{}\\ 
    en Universidad de Burgos --- \@date{}\\
    Tutores: \@tutor{}\\
    		
  }%
  \end{center}%
  \null
  \cleardoublepage
  }
\makeatother

\newcommand{\nombre}{Gonzalo Cuesta Marín} %%% cambio de comando

% Datos de portada
\title{Integración de los datos del CENIEH en \emph{ARIADNEplus}}
\author{\nombre}
\tutor{Dr. Carlos López Nozal\\y D. Mario Juez Gil}
\date{\today}

\begin{document}

\maketitle


\newpage\null\thispagestyle{empty}\newpage


%%%%%%%%%%%%%%%%%%%%%%%%%%%%%%%%%%%%%%%%%%%%%%%%%%%%%%%%%%%%%%%%%%%%%%%%%%%%%%%%%%%%%%%%
\thispagestyle{empty}


\noindent\includegraphics[width=\textwidth]{cabecera}\vspace{1cm}

\noindent D. Carlos López Nozal y D. Mario Juez Gil, profesores del Departamento de Ingeniería Informática, área de Lenguajes y Sistemas Informáticos.

\noindent Exponen:

\noindent Que el alumno D. \nombre, con DNI 71310247N, ha realizado el Trabajo final de Grado en Ingeniería Informática titulado título de TFG. 

\noindent Y que dicho trabajo ha sido realizado por el alumno bajo la dirección del que suscribe, en virtud de lo cual se autoriza su presentación y defensa.

\begin{center} %\large
En Burgos, {\large \today}
\end{center}

\vfill\vfill\vfill

% Author and supervisor
\begin{minipage}{0.45\textwidth}
\begin{flushleft} %\large
Vº. Bº. del Tutor:\\[2cm]
D. Carlos López Nozal
\end{flushleft}
\end{minipage}
\hfill
\begin{minipage}{0.45\textwidth}
\begin{flushleft} %\large
Vº. Bº. del tutor:\\[2cm]
D. Mario Juez Gil
\end{flushleft}
\end{minipage}
\hfill

\vfill


\newpage\null\thispagestyle{empty}\newpage




\frontmatter

% Abstract en castellano
\renewcommand*\abstractname{Resumen}
\begin{abstract}
El Centro Nacional de Investigación sobre la Evolución Humana (CENIEH), se ha incorporado recientemente al proyecto europeo \emph{ARIANDEplus}. Este tiene como objetivo estimular la investigación en áreas relacionadas con la arqueología mediante la integración de diversas infraestructuras de datos arqueológicas situadas en Europa.

Son muchas las técnicas, herramientas y metodologías involucradas en el proceso de integración al que se ven sometidos los datos propuestos por cada uno de los participantes del proyecto. Entre ellas, podemos destacar la estandarización, mapeo y enriquecimiento de metadatos. 

En este trabajo se expone el proceso que se ha llevado a cabo para la integración de los conjuntos de datos del CENIEH en \emph{ARIANDEplus}.

Asímismo, se propone una infraestructura \emph{software} que permite gestionar los conjuntos de datos involucrados en el proceso y provee herramientas que facilitan alguna de las tareas implicadas en el mismo.

\end{abstract}

\renewcommand*\abstractname{Descriptores}
\begin{abstract}
Integración de datos, estandarización e interoperabilidad de datos, mapeo de datos, enriquecimiento de datos, metadatos, sistema de gestión de contenidos.
\end{abstract}

\clearpage

% Abstract en inglés
\renewcommand*\abstractname{Abstract}
\begin{abstract}
The National Research Center for Human Evolution (CENIEH) has recently joined the European project \emph{ARIADNEplus}. This project aims to promote archaeological research by integrating different archaeological databases set in Europe.

There are many methods (and techniques) involved in the integration of data originating from each participating center. Among them, we could highlight Data Standardization, Data Mapping and Data Enrichment. 

This work tries to elaborate on the process followed to integrate the CENIEH data into \emph{ARIADNEplus}. 

We also propose a software-defined infrastructure that allows for the correct management of the included data and provides tools for facilitating some of the tasks involved in the process.
\end{abstract}

\renewcommand*\abstractname{Keywords}
\begin{abstract}
Data integration, data interoperability and standarization, data mapping, data enrichment, metadata, content management system.
\end{abstract}

\clearpage

% Indices
\tableofcontents

\clearpage

\listoffigures

\clearpage

\listoftables
\clearpage

\mainmatter
\capitulo{1}{Introducción}

El CENIEH, Centro Nacional de Investigación sobre la Evolución Humana, es una instalación científica y tecnológica donde se desarrollan multitud de investigaciones relacionadas con la evolución humana. Además, son responsables de la conservación, restauración, gestión y registro de una gran cantidad de colecciones paleontológicas y arqueológicas procedentes de las excavaciones de Atapuerca y otros yacimientos tanto nacionales como internacionales de similares características.

Todas estas actividades generan un gran volumen de datos que, sin el uso de nuevas tecnologías, su almacenamiento sería una tarea verdaderamente compleja. Actualmente disponen de un sistema de base de datos interno que sirve como catálogo de colecciones. Este les permite gestionar la información conceptual (metadatos) de todos los elementos pertenecientes a cada colección como, por ejemplo, su identificación mediante etiquetas RFID y códigos de barras. Además, cuentan con un repositorio online de colecciones llamado CIR (CENIEH Institutional Repository), construido sobre la infraestructura DSpace, donde gestionan una pequeña cantidad de documentos. Al igual que con el sistema anterior, cada documento almacenado es representado como un ítem el cual tiene asignado un conjunto de metadatos y está asociado a una comunidad y colección determinada. 

Exceptuando la información recogida en el CIR, la gran mayoría de datos están almacenados de forma local, es decir, no son accesibles desde el exterior. Esto puede significar un problema ya que cualquier investigador ajeno al CENIEH que pretenda consultar qué documentos tienen en posesión debe personarse en sus instalaciones para llevar a cabo dicha consulta. Como veremos a continuación, AriadnePlus es un proyecto que se presenta como solución a este problema.

AriadnePlus es un proyecto europeo que tiene como finalidad construir una infraestructura de investigación enfocada a la arqueología que fomenta la enseñanza, aprendizaje e investigación a través del acceso a recursos digitales y servicios. El pilar principal de esta infraestructura es su catálogo de colecciones digitales. En él todos los socios del proyecto vuelcan su contenido fruto de investigaciones, excavaciones, trabajos de laboratorio y otros procesos. Además, en el mismo portal donde se encuentra el catálogo, ofrecen multitud de servicios que contribuyen a mejorar la calidad del contenido.

La integración del CENIEH en este proyecto permitirá que todo el contenido almacenado de forma local salga a la red para que investigadores y estudiantes de toda Europa puedan visualizar y acceder a toda esta información de forma remota a través del portal oficial AriadnePlus.

La finalidad de este proyecto será llevar a cabo esta integración. Para ello trataré de diseñar e implementar una infraestructura software que permita gestionar cada uno de los conjuntos de datos que almacena el CENIEH de forma que estos puedan ser importados a AriadnePlus. 



\capitulo{2}{Objetivos del proyecto}

En este apartado se indican, en primer lugar, los objetivos generales
fijados durante el comienzo del proyecto. Seguido de estos, se describen
los objetivos específicos, que se corresponden con los pasos previos que
se han tomado para alcanzar las metas previamente fijadas.

\section{Objetivos generales}\label{obj.gen}

\begin{itemize}
\tightlist
\item
  Integrar los conjuntos de datos propuestos por el CENIEH en
  \emph{ARIADNEplus}.
\item
  Proporcionar al CENIEH una infraestructura \emph{software} que
  permita:

  \begin{quote}
  \begin{itemize}
  \tightlist
  \item
    Gestionar sus (meta)datos en la integración con \emph{ARIADNEplus}.
  \item
    Transformar sus esquemas de (meta)datos a un esquema estandarizado
    compatible con \emph{ARIADNEplus}.
  \item
    Compartir los datos de forma que estos sean accesibles desde el
    exterior.
  \item
    Facilitar la integración de los (meta)datos en \emph{ARIADNEplus}.
  \end{itemize}
  \end{quote}
\end{itemize}

\section{Objetivos específicos}\label{obj.esp}

\begin{itemize}
\tightlist
\item
  Estudiar el proyecto ARIADNEplus, poniendo especial incapié en el
  proceso de integración de los datos.
\item
  Estudiar todos los conjuntos de datos involucrados en el proyecto.
\item
  Diseñar e implementar un esquema de metadatos que satisfaga las
  necesidades de ambas partes, es decir, pueda ser transformado al
  modelo objetivo (\emph{AO-CAT}) y, además, tenga la capacidad de
  representar fehacientemente los conjuntos de datos propuestos por el
  CENIEH.
\item
  Encontrar una aplicación \emph{software} que cumpla con un mínimo de
  requisitos:

  \begin{quote}
  \begin{itemize}
  \tightlist
  \item
    Sea \emph{software} libre.
  \item
    Tolere un esquema de metadatos compatible con \emph{CIDOC-CRM} o
    alguna de sus variantes utilizadas por \emph{ARIADNEplus} como \emph{ACDM}
    o \emph{AO-CAT}.
  \item
    Cuente con un sistema de importación y exportación de metadatos.
  \end{itemize}
  \end{quote}
\item
  Adaptar la aplicación seleccionada a las necesidades del proyecto a
  través del desarrollo de complementos (\emph{plugins}).
\item
  Estudio y uso de \emph{Zend Framework}.
\item
  Aplicar la arquitectura MVC
  (\emph{Model}-\emph{View}-\emph{Controller}) en el desarrollo de los
  \emph{plugins}.
\item
  Estudio y uso de lenguajes empleados para el desarrollo web como
  \emph{PHP}, \emph{HTML}, \emph{JavaScript}, \emph{jQuery} y
  \emph{CSS}.
\item
  Utilizar bases de datos relacionales \emph{MySQL} (\emph{MariaDB}).
\item
  Crear un entorno de desarrollo en *Google Cloud* haciendo uso de *Google Kubernetes Engine*.
\item
  Trabajar con \emph{Docker} para facilitar el despliegue de la
  infraestructura sobre los entornos de trabajo.
\item
  Aplicar técnicas de integración continua a través de herramientas como
  \emph{GitHub Actions} o \emph{Codacy}.
\item
  Aprender a utilizar el conjunto de herramientas alojadas en los entornos de investigación
  virtuales emph{VRES} de la infraestructura \emph{D4Science}, en concreto, \emph{Vocabulary
  Matching Tool} y \emph{X3ML Mapping Tool}.
\item
  Utilizar como sistema de documentación continua \emph{Read the Docs}.
\end{itemize}


\capitulo{3}{Conceptos teóricos}

A lo largo de este apartado se van a exponer los conceptos teóricos
relacionados con las dos primeras fases en las que se divide el
proyecto, que son investigación y desarrollo.

\section{Conceptos teóricos relativos a la investigación}

En esta sección se definen todos aquellos conceptos relacionados con la
investigación previa al desarrollo e implementación de la
infraestructura \emph{software} propuesta.

\subsection{\emph{ARIADNEplus}}

\emph{ARIADNEplus} \cite{arip:web} es la continuación del
proyecto \emph{ARIADNE} \cite{ari:web}, el cual fue fundado
por la Comisión Europea en febrero de 2013. Nació con el propósito de
estimular la investigación en áreas relacionadas con la arqueología
mediante la integración de diversas infraestructuras de datos
arqueológicas situadas en Europa. Fruto de este proyecto surgió un
catálogo \emph{on-line} de (meta)datos referentes a conjuntos de datos
que incluían reportes no publicados, imágenes, mapas, bases de datos, y
otros tipos de información arqueológica.

Este segundo proyecto forma parte del programa \emph{H2020}, fundado
también por la Comisión Europea. El proyecto se encuentra en desarrollo
desde enero de 2019 y tiene previsto una duración total de 48 meses. A
través de \emph{ARIADNEplus}, se actualizarán y extenderán los datos del
catálogo \emph{on-line} anterior añadiendo a los mismos dimensión
geográfica y temporal. 

Además, se van a incorporar más organizaciones
arqueológicas Europeas (entre ellas el CENIEH). También proveerá nuevos
servicios en la nube para procesar y re-utilizar los datos incluidos en
su portal.

\subsection{\emph{CENIEH}}

El Centro Nacional de Investigación sobre la Evolución
Humana \cite{cenieh:web}, tambien conocido como
CENIEH, es una Infraestructura Científica y Técnica Singular (ICTS)
abierta al uso de la comunidad científica y tecnológica, en la que se
desarrollan investigaciones en el ámbito de la evolución humana durante
el Neógeno superior y Cuaternario, promoviendo la sensibilización y
transferencia de conocimientos a la sociedad e impulsando y apoyando la
realización y colaboración en excavaciones de yacimientos de estos
periodos, tanto españoles como de otros países.

Además, el CENIEH es responsable de la conservación, restauración,
gestión y registro de las colecciones paleontológicas y arqueológicas
procedentes de las excavaciones de Atapuerca y otros yacimientos tanto
nacionales como internacionales de similares características.

\subsection{\emph{CIR}}

El CIR \cite{cir:web} (CENIEH Institutional Repository) es el
repositorio bibliográfico institucional del CENIEH. Alberga toda la
información fruto de la actividad investigadora desarrollada en el
CENIEH como, por ejemplo, publicaciones científicas. 

Toda la información que recoge está organizada en ítems que pertenecen 
a una colección, que a su vez forman parte de una comunidad. Cada ítem 
tiene asignado un conjunto de metadatos que describen al objeto digital 
que contiene. El esquema de metadatos utilizado por la plataforma se le 
conoce como \emph{Dublin Core}.

\subsection{Metadatos}

Los metadatos proporcionan la información mínima necesaria para
identificar un recurso, pudiendo incluir información descriptiva sobre
el contexto, calidad y condición o característica del dato \cite{art:meta}. Puede resultar
algo complejo de entender ya que podemos reducir su definición a ``son
datos que describen otros datos''.

Para aportar algo de claridad a esta definición se aplicará el concepto de
``metadato'' tomando como ejemplo una biblioteca. En este contexto, el
conjunto de datos estaría formado por los libros y el conjunto de
metadatos se correspondería con las fichas asociadas a cada libro. Este
ejemplo de metadato está algo anticuado ya que se presenta de una forma
física, no digital.

\imagen{ejemploMetadatos}{Ejemplo de metadatos.}{0.6}

En la actualidad, estas ``fichas'' se encuentran en formato digital a
través de lenguajes de marcado como \emph{XML} o \emph{RDF}.

\subsection{Esquema de metadatos}

Antes de introducir metadatos en cualquier catálogo, es necesario
indicar como van a estar organizados. Para llevar a cabo esta tarea hay
que definir antes un esquema de metadatos, también llamado modelo o
estándar.

Cada esquema está formado por un conjunto de campos de diferentes tipos,
los cuales siguen una estructura jerárquica en forma de árbol.

\imagen{diagramacampos}{Estructura básica de un esquema de metadatos.}{0.7}

En la Figura \ref{fig:diagramacampos}, se muestra la \textbf{estructura básica} 
de cualquier esquema:

\begin{quote}
\begin{itemize}
\item
  \textbf{Ontología}: es la raíz del esquema. Su función es agrupar los
  demás campos en una única unidad temática. Puede tener tres tipos de
  descendientes: Clase, Referencia o Metadato.
\item
  \textbf{Definición de Clase}: define una clase o subclase dentro de
  una ontología determinada, creando así una jerarquía de clases.

  \begin{quote}
  \begin{itemize}
  \tightlist
  \item
    \textbf{Atributo}: define un atributo para una determinada clase
    existente en la ontología.
  \end{itemize}
  \end{quote}
\item
  \textbf{Conjunto de Referencia}: define un conjunto de valores que
  pueden ser instanciados en el Atributo de una Clase o en el Metadato
  de un recurso.

  \begin{quote}
  \begin{itemize}
  \tightlist
  \item
    \textbf{Valor}: define el contenido de cada valor existente en un
    conjunto de referencia.
  \end{itemize}
  \end{quote}
\item
  \textbf{Metadato de Recurso}: define el metadato de un recurso
  determinado. Además, puede ser descendiente de otro metadato a modo de
  especificación.
\end{itemize}
\end{quote}

Cuando se define un atributo o un metadato, se debe indicar, además, el
tipo de contenido que va a adquirir, es decir, señalar qué se va a
introducir. Algunos pueden ser texto plano, otros coordenadas, fechas,
enlaces, etc.

\subsubsection{\emph{CIDOC-CRM}}

\emph{\textbf{CIDOC} \textbf{C}onceptual \textbf{R}eference \textbf{M}odel}
\cite{cidoc:web} (CRM) es una
ontología que ofrece definiciones y una estructura formal para describir
conceptos implícitos y explícitos, así como las relaciones utilizadas en
documentación sobre patrimonio cultural. 

\subsubsection{\emph{ACDM}}

El \emph{\textbf{A}RIADNE \textbf{C}atalogue \textbf{D}ata \textbf{M}odel} es
el modelo de datos utilizado por el catálogo antiguo de \emph{ARIADNE}. Sirve
para describir los recursos arqueológicos publicados por los
participantes del proyecto. El uso de \emph{ACDM} posibilita el descubrimiento,
acceso e integración de los citados recursos. Para formalizar este
modelo, se ha utilizado como base la ontología \emph{CIDOC CRM}, la cual se
adapta correctamente al dominio arqueológico.

\subsubsection{\emph{PEM}}

\emph{PEM} \cite{art:pem}(\emph{\textbf{P}ARTHENOS \textbf{E}ntities \textbf{M}odel}) es el esquema
de metadatos desarrollado en el proyecto \emph{PARTHENOS} \cite{parthenos:web} que extiende al
modelo \emph{CIDOC-CRM}. Está diseñado para ser lo suficientemente flexible
como para mapear los diferentes tipos de esquemas de metadatos
utilizados en todas las disciplinas académicas de manera uniforme.

\subsubsection{AO-Cat}

La ontología \emph{AO-Cat} \cite{art:aocat} (\emph{\textbf{A}RIADNE \textbf{O}ntology
\textbf{-} \textbf{Cat}alog}) deriva del modelo \emph{ACDM}, empleado
por el proyecto antiguo (\emph{ARIADNE}) para modelar recursos arqueológicos, y
del modelo \emph{PEM}, utilizado para modelar cualquier recurso gestionado por
una infraestructura de investigación.

Se podría decir que \emph{AO-Cat} es una
contracción del modelo \emph{ACDM} impulsada por la conceptualización
subyacente al \emph{PEM}. Además, \emph{AO-Cat} hereda del modelo \emph{PEM} su estrecha
relación con el modelo \emph{CIDOC-CRM}, el cual sirve para representar
cualquier aspecto relacionado con recursos arqueológicos.

\emph{AO-Cat} es el \textbf{modelo utilizado por el catálogo actual de ARIADNEplus} y,
por tanto, los metadatos de todos los socios del proyecto se tienen que transformar a este
modelo.

\subsection{Mapeo de datos (\emph{Data Mapping})}

El término ``mapeo'' puede utilizarse en múltiples contextos como, por
ejemplo, en la cartografía, matemáticas, neurociencia, etc. En esta
ocasión, se describirá el concepto relacionado con la informática, más
específicamente con la gestión de datos.

El mapeo de datos consiste en crear asignaciones entre dos elementos que
pertenecen a esquemas de datos distintos. En procesos como la
integración o migración de datos es fundamental llevar a cabo este tipo
de proceso debido a que, generalmente, el sistema al que se trasladan
los datos no utiliza la misma estructura que el sistema de partida.

\imagen{mapping}{Ejemplo de definición de mapeo entre el esquema ``Dublin Core'' y el modelo ``AO-Cat''.}{0.9}

\subsection{Enriquecimiento de datos (\emph{Data Enrichment})}

El enriquecimiento de datos es el proceso mediante el cual es posible
mejorar la calidad de los datos sin necesidad de procesarlos. Durante
este proceso, se fusionan los datos originales con datos de terceros
provenientes de una fuente autorizada externa. 

Para determinar la relación entre los datos originales y los externos se suele hacer uso de
herramientas auxiliares que permiten establecer dichas relaciones.

\imagen{enrichmentconcept}{Proceso de enriquecimiento de datos.}{0.5}

\subsection{\emph{D4Science} -- Entornos de investigación virtuales}

\emph{D4Science} \cite{dfour:web}
es una organización que ofrece una infraestructura de datos basada en
entornos de investigación virtuales (\emph{VREs} \cite{art:vre}). En este tipo de entornos el usuario cuenta con un espacio de trabajo virtual que le da la posibilidad de acceder a datos y compartir los suyos propios. Además, también cuenta con herramientas y capacidad de cómputo
para hacer uso de los datos en su proceso de investigación.


\subsubsection{\emph{ARIADNEplus Gateway}}

\emph{ARIADNEplus} cuenta con un portal en la plataforma
\emph{D4Science} denominado \emph{ARIADNEplus Gateway} \cite{aplusgat:web}. En él tiene
implementados varios entornos virtuales de investigación.
Cada uno de ellos ofrece una serie de servicios que facilitan el proceso
de integración a los miembros del proyecto. Actualmente, cuenta con tres
entornos virtuales, cada uno de los cuales tiene un fin específico:

\imagen{d4scienceVREs}{Entornos virtuales de investigación en D4Science.}{0.9}

\begin{itemize}
\tightlist
\item
  \emph{ARIADNEplus Aggregation Management}: entorno virtual donde
  los líderes del proyecto gestionan las importaciones de metadatos al
  catálogo. El acceso está restringido a los coordinadores del proyecto.
\item
  \emph{ARIADNEplus Mappings}: entorno virtual que da soporte a la
  conversión de metadatos (\emph{mapping}) para su integración en
  \emph{ARIADNEplus}.
\item
  \emph{ARIADNEplus Project}: entorno virtual que permite la
  colaboración y cooperación entre los beneficiarios del proyecto
  \emph{ARIADNEplus}.
\end{itemize}

\subsubsection{\emph{Workspace}}

Otro de los servicios que ofrece \emph{D4Science} es el
\emph{Workspace} \cite{dfourwork:web}. La idea principal de esta herramienta es que los
miembros de un determinado portal intercambien recursos digitales como,
por ejemplo, documentos, imágenes, vídeos, etc.

En este espacio de trabajo los miembros de \emph{ARIADNEplus} organizan y
comparten recursos relacionados con el proyecto como, por ejemplos,
guías, tutoriales, presentaciones, etc.

\imagen{workspace}{Espacio de trabajo (\emph{Workspace}) del proyecto \emph{ARIADNEPlus}.}{1}

Además, este mismo espacio se puede utilizar como medio de importación para el 
catálogo de \emph{ARIADNEPlus}.
Para tal fin, como podemos ver en el imagen, existen dos carpetas
públicas, \emph{Matched Vocabularies} y \emph{Metadata Ingestion}, en
cuyo interior se aloja una carpeta para cada miembro del proyecto. La misión de cada
carpeta es almacenar los ficheros de definición de mapeo de vocabulario (\emph{.json}) y los ficheros con los metadatos (\emph{.xml}). De esta manera, el coordinador puede acceder
a los datos necesarios para llevar a cabo la importación sin necesidad de usar medios externos.

\subsection{\emph{Getty AAT}}

\emph{Getty AAT} \cite{getty:web} es un vocabulario controlado y
estructurado que se emplea para describir elementos de arte,
arquitectura y material cultural. Está compuesto por términos generales
como, por ejemplo, ``Acueducto'', pero no contiene nombres propios como
``Acueducto de Segovia''. Actualmente cuenta con alrededor de 55.000
conceptos registrados, incluyendo 131.000 términos, descripciones,
citaciones bibliográficas, y otra información relacionada con las áreas
previamente mencionadas.

Además cuenta con una interfaz \emph{SPARQL} \cite{getty:sparql} que permite realizar consultas 
sobre los datos (\emph{RDF}) almacenados en su base de datos mediante el lenguaje \emph{SPARQL}.

\subsection{\emph{PeriodO}}

\emph{PeriodO} \cite{getty:web} es un
diccionario digital público donde se almacenan definiciones académicas
de periodos históricos, histórico-artísticos y arqueológicos. Este
proyecto es liderado por Adam Rabinowitz (Universidad de Texas, Austin)
y Ryan Shaw (Universidad de Carolina del norte, Chapel Hill).

\subsection{Tecnología \emph{GraphDB}}

\emph{ARIADNEplus} almacena todos los metadatos en un almacén de \emph{RDF}
(\emph{triplestore}) basado en la tecnología \emph{GraphDB} \cite{gdb:web}. 
Este tipo de tecnología utiliza \textbf{bases de datos orientadas a grafos}. Estas se
basan en un conjunto de objetos (vértices y aristas) que permiten
representar datos interconectados junto a las relaciones existentes
entre sí. 

Cada grafo está compuesto por nodos o vértices, que se
corresponden con los datos (objetos), y aristas o arcos, que serían las
relaciones entre los datos. 

La estructura de este tipo de bases de datos
puede adoptar dos formas: \emph{Labeled-Property Graph} (grafo de
propiedades etiquetadas) o \emph{Resource Description Framework} (marco
de descripción de recursos, \emph{RDF}).

\emph{GraphDB} adopta la segunda estructura, que consiste en estructurar los
grafos mediante \emph{triples} y \emph{quads}: los \emph{triples} están
compuestos por nodo-arco-nodo y los \emph{quads} complementan a estos
con información de contexto adicional, lo que facilita la división de
los datos en grupos. Esta estrutucta es la ideal para almacenar
ontologías como \emph{AO-CAT}, de ahí que \emph{ARIADNEplus} haya escogido esta
tecnología.

\imagen{triple}{\emph{GraphDB} -- \emph{Triple}.}{0.7}

En la Figura \ref{fig:triple} se ha representado un \emph{triple} que se
correspondería con una parte del grafo asociado a la colección \emph{CIR}
almacenada en este tipo de base de datos. Vemos como se compone de dos
nodos, uno para el sujeto (i.e. \emph{CIR}) y otro para el objeto (i.e. \emph{Scientific
analysis}), unidos por un arco, que sería el predicado
(i.e. \emph{has\_ARIADNE\_subject}).

\section{Conceptos teóricos relativos al desarrollo de la infraestructura}

A continuación se definen aquellos conceptos relacionados con el
desarrollo de la infraestructura.

\subsection{Sistema de gestión de contenidos (\emph{CMS})}

Un sistema de gestión de contenidos o \emph{CMS} \cite{wiki:cms} (\emph{\textbf{C}ontent \textbf{M}anagement \textbf{S}ystem}) es una aplicación \emph{software}, generalmente de tipo \emph{web}, que permite crear un entorno de trabajo para la creación y gestión de contenidos. 

Este tipo de sistemas interactúan con una o varias bases de datos que almacenan el contenido sobre el que se realizan las operaciones de gestión. Además, suelen contar con sistemas que permiten adaptar la aplicación, tanto en diseño como en funcionalidad, de una forma sencilla.

La aplicación escogida para este proyecto (\emph{Omeka Classic} \cite{omeka:web}) se puede catalogar como \emph{CMS}.

\subsection{LAMP}

Las siglas \emph{LAMP} \cite{wiki:lamp} son utilizadas para describir infraestructuras
\emph{software} que hacen uso de cuatro herramientas específicas:

\begin{itemize}
\tightlist
\item
  \emph{\textbf{L}inux} como sistema operativo.
\item
  \emph{\textbf{A}pache} como servidor web.
\item
  \emph{\textbf{M}ysql o \textbf{M}ariaDB} como gestor de base de datos.
\item
  \emph{\textbf{P}HP} como lenguaje de programación.
\end{itemize}

La aplicación \emph{software} escogida requiere dicha infraestructura.

\subsection{Complementos (\emph{Plugins})}

Los complementos, más conocidos como \emph{plugins}, son aplicaciones
que permiten ampliar la funcionalidad básica de un determinado producto
software. Normalmente este tipo de aplicaciones son ejecutadas a través
del \emph{software} principal, interactuando con este a través de una
determinada interfaz.

Mediante este tipo de aplicaciones se han conseguido añadir las funcionalidades
requeridas por el proyecto en la aplicación escogida.

\subsection{\emph{Hooking}}

El término \emph{hooking} \cite{wiki:hook} es empleado para referirse a todas aquellas
técnicas utilizadas para modificar el comportamiento de un sistema
operativo, aplicación u otro componente \emph{software} interceptando
llamadas de función, mensajes o eventos pasados entre componentes
\emph{software}. El código que maneja estos acontecimientos se le
denomina \emph{hook}.

\imagen{hooks}{Ejemplo de \emph{hook}.}{0.7}

Haciendo uso de estos \emph{hooks} se ha conseguido modificar el comportamiento
de la aplicación escogida.

\subsection{Prácticas ágiles}

Durante la fase de desarrollo, se han adoptado una serie de prácticas ágiles que han
contribuído favorablemente al desarrollo del \emph{software}. A
continuación, se explica en qué consiste cada una de ellas.

\subsubsection{Desarrollo iterativo e incremental}

En un desarrollo iterativo e incremental el proyecto se va planificando
en intervalos de tiempo constantes, cada uno de los cuales recibe el
nombre de iteración. En todas las iteraciones se sigue un mismo
procedimiento (de ahí el nombre de iterativo) para conseguir una
funcionalidad determinada del producto que se pretende desarrollar.

En cada iteración, se van completando partes del producto final que son
aptas para ser entregadas al cliente. Este goteo constante de entregas
es el responsable de que a este procedimiento se le denomine
incremental. Para que esto sea posible, se definen unos
objetivos/requisitos al inicio de cada iteración que marcarán la
evolución del proyecto. También se pueden plantear mejoras para
requisitos que se entregaron en iteraciones anteriores.

\subsubsection{Pruebas unitarias}

Las pruebas unitarias permiten comprobar el correcto funcionamiento de
unidades de código fuente. Con el uso de este tipo de pruebas se
pretende asegurar que cada unidad se comporta adecuadamente frente a
distintas situaciones. 

Resulta complicado determinar a qué nos referimos
cuando decimos ``unidad de código'' ya que, por definición, se puede asociar
este concepto tanto a una clase como a un método.

Habitualmente se desarrolla más de una prueba unitaria por unidad de
código. El motivo radica en que una prueba unitaria sólo es capaz de
comprobar el comportamiento de la unidad ante una única entrada. Lo
ideal es comprobar su comportamiento ante todas aquellas entradas que
tengan una probabilidad razonable de hacer que falle. El conjunto de
pruebas que recoge todas estas entradas se le denomina \emph{test
suite}.

\subsubsection{Integración y Despliegue continuo (CI/CD)}

La integración continua (\emph{CI}) es una práctica utilizada en el desarrollo
de \emph{software} mediante la cual es posible automatizar operaciones
tales como la compilación o ejecución de pruebas. Aplicando esta
metodologíam se consigue detectar fallos con mayor rapidez, mejorar la
calidad del código y reducir el tiempo empleado en validar y
publicar nuevas actualizaciones \emph{software}.

El despliegue continuo (CD) se puede considerar como el siguiente paso a
la integración continua, es decir, una vez automatizados los procesos de
compilación y ejecución de pruebas, se procede a automatizar el
despliegue del producto \emph{software} que estemos desarrollando.

\section{Otros conceptos}

En este apartado se recogen todos aquellos conceptos que tienen cierta
relevancia en el proyecto y no han sido expuestos en secciones
anteriores.

\subsection{\emph{Dublin Core}}

\emph{Dublin Core} es un esquema de metadatos elaborado por la
\emph{DCMI} \cite{dc:web}, organización cuya misión
principal es facilitar la compartición de recursos \emph{on-line} por
medio del desarrollo de un modelo de metadatos ``base'', capaz de
proporcionar información descriptiva básica sobre cualquier recurso, sin
importar el formato de origen, área de especialización u origen
cultural. 

Dispone de 15 elementos descriptivos, los cuales pueden ser
repetidos, aparecer en cualquier orden y estar o no presentes
(son opcionales).

\subsection{\emph{Dublin Core Extended}}

Dado que el modelo \emph{Dublin Core} puede resultar algo escueto, se
presenta como solución el esquema \emph{Dublin Core Extended}, el cual
cuenta con los elementos descriptivos del modelo original y, además,
incluye una serie de elementos adicionales/complementarios \cite{dcterms:web}
que satisfacen las necesidades que el modelo original no cubre.

Este modelo ha sido el propuesto para transformar todos los conjuntos de 
datos del \emph{CENIEH} a un único modelo estándar.

\subsection{Interoperabilidad}

La interoperabilidad es la capacidad que tiene un sistema o producto de
compartir datos y posibilitar el intercambio de información y
conocimiento entre ellos \cite{interop:web}.
En lo que respecta a repositorios, se puede conseguir dicha capacidad
haciendo uso de estándares como, por ejemplo, el protocolo
\emph{OAI-PMH}.

\subsection{Protocolo \emph{OAI-PMH}}

El protocolo \emph{Open Archive Initiative-Protocol for Metadata
Harvesting} (\emph{OAI-PMH}) tiene como objetivo desarrollar y promover
estándares de interoperabilidad que faciliten la difusión eficiente de
contenidos en Internet. Permite transmitir metadatos entre diferentes
tipos de infraestructuras \emph{software} (repositorios, gestores, etc.)
siempre y cuando éstos se codifiquen en \emph{Dublin Core}.

Gracias a que la aplicación escogida ofrece este servicio, haciendo uso
del mismo se han podido recolectar todos los metadatos existentes en el
\emph{CIR}. Además, \emph{ARIADNEplus} permite importar metadatos en su catálogo
haciendo uso de este protocolo, por lo que su implantación también abre
otro posible camino de importación.

\imagen{oai-pmh}{Ejemplo básico del protocolo \emph{OAI-PMH}.}{0.5}

\subsection{Geolocalización}

La geolocalización es la capacidad para obtener la ubicación geográfica
real de un objeto \cite{wiki:geo}. Uno de
los requisitos fundamentales del catálodo de \emph{ARIADNEplus} es que todos
los metadatos importados han de estar geolocalizados, es decir, tienen
que tener, al menos, un elemento descriptivo que indique la ubicación
actual del objeto. Nuestra plataforma cuenta con el elemento
\emph{Spatial Coverage} del modelo \emph{Dublin Core Extended} para
cubrir este requisito.

\subsubsection{WSG84}

El \textbf{W}orld \textbf{G}eodetic \textbf{S}ystem \textbf{84} es un
sistema de coordenadas geográficas usado mundialmente para localizar
cualquier punto de la Tierra \cite{wiki:wsg}. Uno de los requisitos del catálogo de
\emph{ARIADNEplus} es que todas aquellas localizaciones señaladas a través de
coordenadas geográficas deben utilizar este sistema.


\capitulo{4}{Técnicas y herramientas}

\section{Metodologías}\label{metodologias}

\subsection{Scrum}\label{scrum}

Scrum es un marco de trabajo donde se ejecutan procesos ágiles que contribuyen al desarrollo y mantenimiento de productos \emph{software}. Por ello, está catalogado como una metodología ágil, la cual se caracteriza por trabajar con un ciclo de vida iterativo e incremental, donde se va liberando el producto software de forma periódica a través de \emph{sprints} (iteraciones) \cite{agile:scrum}.

\section{Cliente de control de versiones}\label{ctr-ver}
\begin{itemize}
\tightlist
	\item Herramientas consideradas:
 		\href{https://wiki.gnome.org/Apps/Gitg/}{Gitg},
  		\href{https://www.syntevo.com/smartgit/}{SmartGit},
  		\href{https://www.gitkraken.com/}{GitKraken} y
  		\href{https://desktop.github.com/}{GitHub Desktop}
	\item Herramienta elegida: \href{https://www.gitkraken.com/}{GitKraken}.
\end{itemize}

GitKraken es un cliente para el control de versiones \emph{Git} que nos permite realizar todas y cada una de las tareas propias de \emph{Git} a través de una intuitiva y elegante interfaz gráfica. Además, incorpora funciones adicionales como \emph{GitFlow}, que nos permite gestionar las diferentes ramificaciones del proyecto. De entre todas las opciones posibles es, en mi opinión, la más competente tanto en diseño como en funcionalidad.

\section{\emph{Hosting} del repositorio}\label{rep-host}
\begin{itemize}
\tightlist
	\item Herramientas consideradas: \href{https://github.com/}{GitHub} y
  		\href{https://gitlab.com/}{GitLab}.
	\item Herramienta elegida: \href{https://github.com/}{GitHub}.
\end{itemize}

Github es el servicio de \emph{hosting} de Git más utilizado para albergar repositorios de código en la nube. El hecho de que cuente con una enorme comunidad de usuarios y que además ofrezca servicios exclusivos y gratuitos a estudiantes lo convierte en la mejor opción posible. Alguno de estos servicios son: privatización de repositorios, cantidad ilimitada de colaboradores por repositorio, código propiertario... 

\section{Gestor de contenidos (CMS)}\label{cms}

\begin{itemize}
\tightlist
	\item Gestores de contenidos considerados:
  		\href{https://duraspace.org/dspace/}{DSpace},
  		\href{https://www.bibl.ulaval.ca/archimede/index.en.html}{Archimede},
  		\href{https://www.mycore.de/}{MyCoRe},
  		\href{https://omeka.org/classic/}{Omeka Classic} y
  		\href{https://github.com/DANS-KNAW/dccd-webui}{DCCD}
	\item Herramienta elegida: \href{https://omeka.org/classic/}{Omeka Classic}.
\end{itemize}

Omeka Classic es una plataforma de gestión de contenidos libre, flexible y de código abierto. Su misión principal es la publicación de colecciones digitales provenientes de bibliotecas, museos o cualquier tipo de institución que pretenda difundir su patrimonio cultural, como es el caso del CENIEH. Los motivos principales por los que he decidido escoger este CMS son:
\begin{enumerate}
\tightlist
	\item Se distribuye bajo una Licencia Pública General1 (GNU), con lo cual su distribución, uso y modificación es libre. Esto me permitirá amoldar la plataforma a los requisitos impuestos por la integración.
	\item Utiliza un entorno PHP-MySQL (Fácil despliegue sobre el servidor)
	\item Basado en estándares internacionalmente aceptados como Dublin Core o W3C. 
	\item Flexible, escalable y extensible.
	\begin{itemize}
	\tightlist
		\item Zend framework como arquitectura.
		\item APIs documentadas.
		\item Le respalda una gran comunidad de desarrolladores.
	\end{itemize}
	\item Asistencia técnica gratuita gracias a la existencia de foros donde desarrolladores del proyecto oficial aportan soluciones.
	\item Pensado para ser utilizado por usuarios no necesariamente expertos en el manejo de las TIC (personal del CENIEH).
\end{enumerate}

\section{Entorno de desarrollo integrado (IDE)}\label{ide}

\subsection{PHP | CSS | JavaScript | XML}\label{java}

\begin{itemize}
\tightlist
	\item Herramientas consideradas:
  		\href{https://atom.io/}{Atom},
  		\href{https://eclipse.org/}{Eclipse},
  		\href{https://www.zend.com/products/zend-studio}{Zend Studio} y
  		\href{https://www.activestate.com/products/komodo-ide/}{Komodo}  
	\item Herramienta elegida: \href{https://www.zend.com/products/zend-studio}{Zend Studio}.
\end{itemize}

Zend Studio es un IDE para PHP que ha sido construido tomando como base Eclipse. Considero que es la opción ideal ya que da soporte a todos y cada uno de los lenguajes de programación utilizados por la infraestructura software escogida. Además, permite la instalación de \emph{plugins} externos (Eclipse) e incluye herramientas tales como \emph{Docker} y \emph{Gitflow}. 

\subsection{LaTeX}\label{latex}

\begin{itemize}
\tightlist
	\item Herramientas consideradas:
  		\href{https://www.texstudio.org/}{TeXstudio} y
  		\href{http://www.xm1math.net/texmaker/}{Texmaker}.
	\item Herramienta elegida:
  		\href{http://www.xm1math.net/texmaker/}{Texmaker}.
\end{itemize}

Texmaker es un editor libre y gratuito para \LaTeX distribuido bajo la licencia GPL. Además, es multi-plataforma, es decir, trabaja tanto en UNIX como en MacOS y Windows. Incluye múltiples herramientas necesarias para elaborar documentos con \LaTeX como BibText o Metapost. Incorpora funciones adicionales como la corrección ortográfica, el auto-completado y plegado de código o un visor de pdf compatible con SyncTeX y con modo de visualización continua. 

\section{Generador de documentación}\label{gen-doc}

\begin{itemize}
\tightlist
	\item Herramientas consideradas:
  		\href{https://www.sphinx-doc.org/es/master/index.html}{Sphinx} y
  		\href{https://www.mkdocs.org/} {MkDocs}
	\item Herramienta elegida:
  		\href{https://www.sphinx-doc.org/es/master/index.html}{Sphinx}
\end{itemize}

He decidido utilizar el generador de documentación Sphinx ya que es mucho más completo que MkDocs. Además de soportar el lenguaje de marcado ligero \emph{Markdown} es compatible con \emph{reStructuredText}. Esta compatibilidad hace que sea posible usar ambos lenguajes en un mismo proyecto Sphinx. Además, con el uso del conversor \href{http://pandoc.org/}{Pandoc}, toda la documentación generada a partir de ambos lenguajes se puede exportar a multitud de formatos, entre los que se encuentra \LaTeX.

\emph{Markdown} es un lenguaje muy conocido debido a que es utilizado en plataformas como Github o StackOverflow. Fue creado para generar contenido de una manera sencilla de escribir y fácil de leer. Permite además convertir el texto marcado en documentos XHTML.

\emph{reStructuredText} presenta también una sintaxis sencilla y de fácil lectura. La principal ventaja respecto a \emph{Markdown} es que permite elaborar expresiones más complejas sin el uso de librerías/aplicaciones externas.

\LaTeX es el estándar de facto para la publicación de documentos científicos. Permite la creación de documentos con una alta calidad tipográfica. Utiliza Tex como motor a la hora de darle formato a los documentos.
% Options for packages loaded elsewhere
\PassOptionsToPackage{unicode}{hyperref}
\PassOptionsToPackage{hyphens}{url}
%
\documentclass[
]{article}
\usepackage{lmodern}
\usepackage{amssymb,amsmath}
\usepackage{ifxetex,ifluatex}
\ifnum 0\ifxetex 1\fi\ifluatex 1\fi=0 % if pdftex
  \usepackage[T1]{fontenc}
  \usepackage[utf8]{inputenc}
  \usepackage{textcomp} % provide euro and other symbols
\else % if luatex or xetex
  \usepackage{unicode-math}
  \defaultfontfeatures{Scale=MatchLowercase}
  \defaultfontfeatures[\rmfamily]{Ligatures=TeX,Scale=1}
\fi
% Use upquote if available, for straight quotes in verbatim environments
\IfFileExists{upquote.sty}{\usepackage{upquote}}{}
\IfFileExists{microtype.sty}{% use microtype if available
  \usepackage[]{microtype}
  \UseMicrotypeSet[protrusion]{basicmath} % disable protrusion for tt fonts
}{}
\makeatletter
\@ifundefined{KOMAClassName}{% if non-KOMA class
  \IfFileExists{parskip.sty}{%
    \usepackage{parskip}
  }{% else
    \setlength{\parindent}{0pt}
    \setlength{\parskip}{6pt plus 2pt minus 1pt}}
}{% if KOMA class
  \KOMAoptions{parskip=half}}
\makeatother
\usepackage{xcolor}
\IfFileExists{xurl.sty}{\usepackage{xurl}}{} % add URL line breaks if available
\IfFileExists{bookmark.sty}{\usepackage{bookmark}}{\usepackage{hyperref}}
\hypersetup{
  pdftitle={Aspectos relevantes del desarrollo del proyecto},
  hidelinks,
  pdfcreator={LaTeX via pandoc}}
\urlstyle{same} % disable monospaced font for URLs
\usepackage{longtable,booktabs}
% Correct order of tables after \paragraph or \subparagraph
\usepackage{etoolbox}
\makeatletter
\patchcmd\longtable{\par}{\if@noskipsec\mbox{}\fi\par}{}{}
\makeatother
% Allow footnotes in longtable head/foot
\IfFileExists{footnotehyper.sty}{\usepackage{footnotehyper}}{\usepackage{footnote}}
\makesavenoteenv{longtable}
\usepackage{graphicx}
\makeatletter
\def\maxwidth{\ifdim\Gin@nat@width>\linewidth\linewidth\else\Gin@nat@width\fi}
\def\maxheight{\ifdim\Gin@nat@height>\textheight\textheight\else\Gin@nat@height\fi}
\makeatother
% Scale images if necessary, so that they will not overflow the page
% margins by default, and it is still possible to overwrite the defaults
% using explicit options in \includegraphics[width, height, ...]{}
\setkeys{Gin}{width=\maxwidth,height=\maxheight,keepaspectratio}
% Set default figure placement to htbp
\makeatletter
\def\fps@figure{htbp}
\makeatother
\setlength{\emergencystretch}{3em} % prevent overfull lines
\providecommand{\tightlist}{%
  \setlength{\itemsep}{0pt}\setlength{\parskip}{0pt}}
\setcounter{secnumdepth}{-\maxdimen} % remove section numbering

\title{Aspectos relevantes del desarrollo del proyecto}
\author{}
\date{}

\begin{document}
\maketitle

A continuación se van a mostrar los aspectos más relevantes del
desarrollo del proyecto, justificando las decisiones tomadas y dando
visibilidad a los problemas encontrados.

\hypertarget{ciclo-de-vida-del-proyecto}{%
\section{Ciclo de vida del proyecto}\label{ciclo-de-vida-del-proyecto}}

El 01 de febrero de 2020 se inició este proyecto con el objetivo de
llevar a cabo el proceso de integración de los datos del CENIEH en
ARIADNEplus. Adicionalmente, se propuso implantar en el CENIEH una
infraestructura \emph{software} mediante la cual los investigadores del
CENIEH fueran capaces de llevar a cabo las tareas de integración por sí
solos. De esta manera, a través de la infraestructura propuesta, serían
capaces de continuar con las labores de integración una vez concluido el
periodo de colaboración entre el CENIEH y la UBU.

El proyecto ha tenido una duración total de 5 meses, lo que significa
que el 01 de julio de 2020 finalizaron las tareas implicadas en el
proyecto. Sin embargo, las tareas de integración se siguieron llevando a
cabo desde el CENIEH hasta el 31 de julio de 2020, fecha en la que se da
por concluido el periodo de colaboración.

Las fases en las que se puede dividir el proyecto son:

\begin{itemize}
\tightlist
\item
  \textbf{Investigación}: en esta fase se realiza un estudio previo del
  proyecto ARIADNEplus, así como de los conjuntos de datos del CENIEH
  que están involucrados en el proceso de integración.
\item
  \textbf{Desarrollo}: a lo largo de esta fase se ejecutan todas las
  tareas relacionadas con el diseño, desarrollo e implementación de la
  infraestructura \emph{software}.
\item
  \textbf{Integración}: en esta fase se integran los datos del CENIEH en
  ARIADNEplus haciendo uso tanto de la infraestructura \emph{software}
  implementada como de los servicios ofrecidos por ARIADNEplus.
\end{itemize}

\hypertarget{investigaciuxf3n}{%
\section{Investigación}\label{investigaciuxf3n}}

Durante las primeras semanas de trabajo, se llevó a cabo un estudio
exhaustivo del proyecto ARIADNEplus, especialmente del \textbf{proceso
de integración} al que cada miembro debía someter sus datos. A medida
que se iban aprendiendo nuevos aspectos, se comprobaba su compatibilidad
con los datos propuestos por el CENIEH, anotando en todo momento los
problemas que pudieran surgir.

\hypertarget{proceso-de-integraciuxf3n}{%
\subsection{Proceso de integración}\label{proceso-de-integraciuxf3n}}

ARIADNEplus no agrega ni mueve datos de los sistemas de información de
los miembros del proyecto, solo añade \textbf{metadatos} a los conjuntos
de datos que son mantenidos y gestionados por cada miembro. Este tipo de
información debe adoptar un \textbf{esquema o modelo} para poder ser
representado a través de catálogos o repositorios. En el caso de
ARIADNEplus, cuentan con un esquema denominado \textbf{AO-Cat}
(\textbf{A}RIADNE \textbf{O}ntology \textbf{-} \textbf{Cat}alog),
diseñado exclusivamente para el proyecto.

Cada miembro del proyecto, incluído el CENIEH, cuenta desde un principio
con sus propias colecciones de metadatos. Esto puede considerarse un
problema ya que, como se ha comentado anteriormente, este tipo de
información se representa a través de un esquema, el cual será distinto
en cada uno de los miembros. Ante esta situación, al no coincidir el
esquema de origen (?) con el de destino (AO-Cat), sería imposible poder
representar los datos del origen en el catálogo.

\begin{figure}
\hypertarget{mappingProblem}{%
\centering
\includegraphics{../_static/images/mappingProblem.png}
\caption{Conflicto entre esquemas de metadatos
distintos}\label{mappingProblem}
}
\end{figure}

Además, un esquema está formado por un conjunto de elementos, cada uno
de los cuales está sujeto a unas determinadas reglas. Por ejemplo,
podría especificarse el tipo de dato que almacena (\emph{string},
\emph{date}, etc.) o considerarse como imprescindible (no nulo). Este
hecho aumenta aún más la complejidad del problema de integración ya que
cada miembro cuenta con sus propios elementos y con sus propias reglas.

Para dar solución a todos estos problemas, ARIADNEplus propone dividir
el proceso de integración en 6 fases.

\begin{figure}
\hypertarget{integrationPhases}{%
\centering
\includegraphics{../_static/images/integrationPhases.png}
\caption{Fases en las que se divide el proceso de
integración}\label{integrationPhases}
}
\end{figure}

\begin{enumerate}
\def\labelenumi{\arabic{enumi}.}
\tightlist
\item
  \textbf{Confirmación}: se confirman las colecciones de metadatos que
  serán agregadas y, además, se indica a qué categoria de datos de
  ARIADNEplus pertenecen.
\item
  \textbf{Transformación}: una vez estén listos los metadatos de origen,
  se genera un fichero de definición de mapeo que permita transformar el
  esquema de metadatos de origen al esquema objetivo (AO-Cat).
\item
  \textbf{Enriquecimiento}: se mejora la calidad de los metadatos
  sometiendo los datos a un proceso de enriquecimiento utilizando el
  vocabulario \emph{Getty AAT} y/o \emph{PeriodO}.
\item
  \textbf{Importación}: habiendo completado las tres fases anteriores,
  se ejecuta el proceso de importación de los metadatos.
\item
  \textbf{Simulación de publicación}: con los metadatos de origen ya
  importados en la base de datos, se realiza una simulación de
  publicación.
\item
  \textbf{Publicación}: si los resultados obtenidos en la fase 5 son
  favorables, se lleva a cabo la publicación de los metadatos en el
  catálogo oficial.
\end{enumerate}

\hypertarget{confirmaciuxf3n}{%
\subsubsection{Confirmación}\label{confirmaciuxf3n}}

En esta fase se confirman qué colecciones de datos serán integradas en
el proyecto. Además, se debe indicar a qué categoría de datos de ARIADNE
pertenecen.

\begin{figure}
\hypertarget{categoriesARIADNE}{%
\centering
\includegraphics{../_static/images/categoriesARIADNE.png}
\caption{Fase de confirmación}\label{categoriesARIADNE}
}
\end{figure}

Afortunadamente, los datos propuestos se obtuvieron con suma facilidad,
sin ningún tipo de reticencia por parte del CENIEH. En la siguiente
tabla se indican las principales características de cada una de las
colecciones de datos confirmadas para la integración con ARIADNEplus.
Además, se especifica la categoría ARIADNE a la que corresponden.

\begin{longtable}[]{@{}llllll@{}}
\caption{Colecciones de metadatos propuestas por el CENIEH para la
integración con ARIADNEplus.}\tabularnewline
\toprule
Colección & Núm. Registros & Docs asociados & Campos & Formato &
Categoría ARIADNE\tabularnewline
\midrule
\endfirsthead
\toprule
Colección & Núm. Registros & Docs asociados & Campos & Formato &
Categoría ARIADNE\tabularnewline
\midrule
\endhead
\begin{minipage}[t]{0.14\columnwidth}\raggedright
Anatomía Comparada\strut
\end{minipage} & \begin{minipage}[t]{0.14\columnwidth}\raggedright
\begin{quote}
571
\end{quote}\strut
\end{minipage} & \begin{minipage}[t]{0.14\columnwidth}\raggedright
\begin{quote}
Sí
\end{quote}\strut
\end{minipage} & \begin{minipage}[t]{0.14\columnwidth}\raggedright
SIGNA\_CENIEH, Clase, Orden, Familia Género, Especie, Sigla de campo,
Elemento, Sexo, Adulto Localidad, Municipio, Provincia, Pais, Tipo de
objeto\strut
\end{minipage} & \begin{minipage}[t]{0.14\columnwidth}\raggedright
\begin{quote}
CSV
\end{quote}\strut
\end{minipage} & \begin{minipage}[t]{0.14\columnwidth}\raggedright
Scientific analysis\strut
\end{minipage}\tabularnewline
\begin{minipage}[t]{0.14\columnwidth}\raggedright
\begin{quote}
Litoteca
\end{quote}\strut
\end{minipage} & \begin{minipage}[t]{0.14\columnwidth}\raggedright
\begin{quote}
99
\end{quote}\strut
\end{minipage} & \begin{minipage}[t]{0.14\columnwidth}\raggedright
\begin{quote}
Sí
\end{quote}\strut
\end{minipage} & \begin{minipage}[t]{0.14\columnwidth}\raggedright
Afloramiento, Sigla, Localización, Datum, X, Y, Z, Acceso, Tipo de
Afloramiento, Tipo de roca, Depositante, Muestra física, Lámina delgada,
Laboratorio geología CENIEH, Fotografías, Otros datos, Topografía\strut
\end{minipage} & \begin{minipage}[t]{0.14\columnwidth}\raggedright
\begin{quote}
CSV
\end{quote}\strut
\end{minipage} & \begin{minipage}[t]{0.14\columnwidth}\raggedright
Scientific analysis\strut
\end{minipage}\tabularnewline
\begin{minipage}[t]{0.14\columnwidth}\raggedright
\begin{quote}
Ratón Pérez
\end{quote}\strut
\end{minipage} & \begin{minipage}[t]{0.14\columnwidth}\raggedright
\begin{quote}
1323
\end{quote}\strut
\end{minipage} & \begin{minipage}[t]{0.14\columnwidth}\raggedright
\begin{quote}
Sí
\end{quote}\strut
\end{minipage} & \begin{minipage}[t]{0.14\columnwidth}\raggedright
Sigla, Individuo, Sexo, Edad, Pieza, Superior/inferior, Lado,
Conservación, Consolidado, Pegado, Observaciones, Localización, Fecha
MicroCT, Archivo mCT, Proyecto Amira, localización, No Imágenes, kv/mA,
Vxl, Size, Filter, Fotos mCT\strut
\end{minipage} & \begin{minipage}[t]{0.14\columnwidth}\raggedright
\begin{quote}
CSV
\end{quote}\strut
\end{minipage} & \begin{minipage}[t]{0.14\columnwidth}\raggedright
Scientific analysis\strut
\end{minipage}\tabularnewline
\begin{minipage}[t]{0.14\columnwidth}\raggedright
\begin{quote}
Sedimentos
\end{quote}\strut
\end{minipage} & \begin{minipage}[t]{0.14\columnwidth}\raggedright
\begin{quote}
7695
\end{quote}\strut
\end{minipage} & \begin{minipage}[t]{0.14\columnwidth}\raggedright
\begin{quote}
No
\end{quote}\strut
\end{minipage} & \begin{minipage}[t]{0.14\columnwidth}\raggedright
ReferenciaBolsa, ReferenciaCaja, Yacimiento, Nivel, Cuadro, Z,
Situacion, FechaRecogida, FechaAlmacen, FechaProcesando\strut
\end{minipage} & \begin{minipage}[t]{0.14\columnwidth}\raggedright
\begin{quote}
CSV
\end{quote}\strut
\end{minipage} & \begin{minipage}[t]{0.14\columnwidth}\raggedright
Scientific analysis\strut
\end{minipage}\tabularnewline
\begin{minipage}[t]{0.14\columnwidth}\raggedright
\begin{quote}
CIR
\end{quote}\strut
\end{minipage} & \begin{minipage}[t]{0.14\columnwidth}\raggedright
\begin{quote}
1853
\end{quote}\strut
\end{minipage} & \begin{minipage}[t]{0.14\columnwidth}\raggedright
\begin{quote}
Sí
\end{quote}\strut
\end{minipage} & \begin{minipage}[t]{0.14\columnwidth}\raggedright
\emph{Dublin Core terms}\strut
\end{minipage} & \begin{minipage}[t]{0.14\columnwidth}\raggedright
\begin{quote}
CSV
\end{quote}\strut
\end{minipage} & \begin{minipage}[t]{0.14\columnwidth}\raggedright
Scientific analysis\strut
\end{minipage}\tabularnewline
\bottomrule
\end{longtable}

\hypertarget{transformaciuxf3n}{%
\subsubsection{Transformación}\label{transformaciuxf3n}}

Para evitar el problema mostrado en la \texttt{mappingProblem},
ARIADNEplus pone a disposición de sus miembros la \textbf{herramienta
X3ML Mapping Tool}, disponible en el VRE \emph{ARIADNEplus Mappings} del
portal \emph{ARIADNEplus Gateway} de \emph{D4Science}. Está compuesta
por un conjunto de microservicios, de código abierto, que siguen el
modelo de referencia \emph{SYNERGY} para la transmisión y agregación de
datos.

\begin{figure}
\hypertarget{transformARIADNE}{%
\centering
\includegraphics{../_static/images/transformARIADNE.png}
\caption{Fase de transformación}\label{transformARIADNE}
}
\end{figure}

Los componentes clave de este servicio son:

\begin{itemize}
\tightlist
\item
  \emph{3M -- Mapping Memory Manager}: herramienta utilizada para la
  gestión de archivos de definición de mapeo. Proporciona una serie de
  acciones administrativas que ayudan a los proveedores de datos a
  administrar sus archivos de definición de mapeo.
\end{itemize}

\begin{figure}
\hypertarget{mmm3m}{%
\centering
\includegraphics{../_static/images/mmm3m.png}
\caption{Vista de la herramienta \emph{Mapping Memory Manager -
3M}}\label{mmm3m}
}
\end{figure}

\begin{itemize}
\tightlist
\item
  \emph{3M Editor}: provee la interfaz que permite crear asignaciones
  entre los elementos del esquema de metadatos a mapear y el esquema
  objetivo.
\end{itemize}

\begin{figure}
\hypertarget{3meditor}{%
\centering
\includegraphics{../_static/images/3meditor.png}
\caption{Vista de la herramienta \emph{3M Editor}}\label{3meditor}
}
\end{figure}

\begin{itemize}
\tightlist
\item
  \emph{X3ML Engine}: ejecuta la transformación de los elementos de
  origen al formato de destino. Tomando como entrada los datos de origen
  (en formato XML), la descripción de las asignaciones existentes en el
  fichero de definición de mapeo y el archivo que contiene las políticas
  para la generación de URIs, es responsable de transformar el documento
  original en un documento RDF válido que corresponda al archivo XML de
  entrada con las asignaciones y políticas indicadas.
\item
  \emph{RDF visualizer}: permite, de una forma rápida, inspeccionar los
  documentos transformados.
\end{itemize}

\begin{figure}
\hypertarget{rdfvisualizer}{%
\centering
\includegraphics{../_static/images/rdfvisualizer.png}
\caption{Vista de la herramienta \emph{RDF
visualizer}}\label{rdfvisualizer}
}
\end{figure}

Esta herramienta toma un \textbf{papel decisivo} en el proceso de
integración ya que permite transformar el modelo de origen al esquema de
metadatos utilizado en ARIADNEplus (AO-CAT).

A continuación se van a describir los principales \textbf{retos} a los
que nos hemos tenido que enfrentar durante esta segunda fase:

\begin{itemize}
\tightlist
\item
  Todos los conjuntos de datos propuestos por el CENIEH están en formato
  CSV. Esto supone un problema ya que \textbf{ARIADNEplus solo trabaja
  con ficheros XML}, es decir, no cuenta con ningún método de
  importación que tolere archivos CSV.
\item
  \textbf{Los conjuntos de datos del CENIEH}, a excepción de la
  colección del CIR, están dispuestos de forma irregular, es decir,
  \textbf{no siguen ningún esquema estandarizado}. Esto implica que para
  cada conjunto de datos, se necesita hacer un fichero de definición de
  mapeo distinto, lo que no es para nada eficiente.
\item
  En el esquema objetivo, los \textbf{elementos} pueden ser opcionales u
  \textbf{obligatorios}. Los elementos opcionales no suponen ningún
  problema ya que pueden quedar vacíos, sin embargo, los elementos
  obligatorios requieren la existencia de un elemento en el modelo de
  origen que pueda sustituirlo, es decir, que tenga el mismo
  significado. Esta regla supone un reto para el CENIEH ya que muchos de
  los elementos obligatorios no cuentan con un elemento apto en las
  colecciones de datos propuestas.
\item
  \textbf{El contenido} almacenado en cada elemento del esquema objetivo
  \textbf{ha de tener un formato específico}. Por ejemplo, el contenido
  del elemento \emph{has\_language}, responsable de indicar el idioma en
  el que está dispuesto el objeto al que referencia, debe cumplir con el
  estándar ISO639-1 o ISO639-2. Por tanto, el elemento asignado en el
  origen debe seguir el mismo formato.
\end{itemize}

\hypertarget{enriquecimiento}{%
\subsubsection{Enriquecimiento}\label{enriquecimiento}}

En ocasiones, los metadatos por si solos no son lo suficientemente
precisos o claros como para describir una determinada característica del
objeto al que se refieren. En el caso de la arqueología, existen
multitud de conceptos con un alto grado de complejidad que necesitan ser
explicados en detalle. Por este motivo, ARIADNEplus propone enriquecer
los metadatos haciendo uso del vocabulario \emph{Getty AAT} y del
cliente \emph{PeriodO}.

\begin{figure}
\hypertarget{enrichment}{%
\centering
\includegraphics{../_static/images/enrichment.png}
\caption{Enriquecimiento de metadatos}\label{enrichment}
}
\end{figure}

En la \texttt{enrichment} se muestra el flujo de datos del proceso de
enriquecimiento de metadatos. Por una parte, vemos un archivo .json, el
cual se obtiene a través de la \textbf{herramienta Vocabulary Matching
Tool}. Esta es otra de las herramientas que se pueden encontrar en el
VRE \emph{ARIADNEplus Mappings}. Permite mapear el vocabulario utilizado
en el documento de origen al vocabulario \emph{Getty AAT}.

\begin{figure}
\hypertarget{vmt}{%
\centering
\includegraphics{../_static/images/vmt.png}
\caption{Vista de la herramienta \emph{Vocabulary Matching
Tool}}\label{vmt}
}
\end{figure}

El archivo generado por esta herramienta (.json) define las relaciones
entre los conceptos del vocabulario de origen y los conceptos del
vocabulario \emph{Getty AAT}. Desde el catálogo de ARIADNEplus, todos
aquellos términos que tengan una asociación definida, serán hiperenlaces
que apunten al concepto Getty AAT asociado.

Además, vemos representada la BD donde \textbf{PeriodO} almacena sus
registros. Para aportar información adicional a los periodos existentes
en nuestros datos, debemos publicar en el cliente de PeriodO nuestra
propia colección de periodos. De esta forma, ARIADNEplus podrá recoger
desde la BD de periodO nuestra colección para, posteriormente,
establecer una relación entre los periodos de un lado y de otro. Al
igual que con el vocabulario, todos los periodos que tengan una
asociación definida, serán hiperenlaces que apunten al objeto de
periodO.

En esta fase se encontraron varios \textbf{inconvenientes}: - Muchos de
los términos existentes en los conjuntos de datos del CENIEH no están
presentes en el vocabulario \emph{Getty AAT}. Por este motivo, solo se
pudo enriquecer una pequeña parte del conjunto total. - Para poder
publicar la colección en periodO, se requería determinar la autoridad de
los periodos, es decir, indicar de donde procedían. Desde el CENIEH no
me pudieron facilitar ese dato ya que lo desconocían. Por este motivo,
no se pudo llevar a cabo la publicación y por ende no se enriquecerieron
los periodos.

\hypertarget{importaciuxf3n}{%
\subsubsection{Importación}\label{importaciuxf3n}}

El sistema de importación de ARIADNEplus, conocido como ARIADNEplus
\emph{Aggregator}, se basa en el kit de herramientas de \emph{software}
D-Net (implementado y mantenido por ISTI-CNR\footnote{"ISTI-CNR --
  Istituto di Scienza e Tecnologie dell'Informazione "
  \url{https://www.isti.cnr.it/}}), que proporciona funciones integradas
que permiten recopilar conjuntos de metadatos a través de múltiples
métodos. Está disponible en el portal \emph{ARIADNEplus Gateway}, sin
embargo, su acceso está restringido a los coordinadores del proyecto.
Las principales opciones son:

\begin{enumerate}
\def\labelenumi{\arabic{enumi}.}
\tightlist
\item
  \textbf{OAI-PMH}: es un protocolo estándar para el intercambio de
  metadatos. A través de este método ARIADNEplus puede recopilar todo el
  contenido o los conjuntos de datos OAI que le indiquemos.
\item
  \textbf{SFTP}: es un protocolo de transmisión de ficheros. Esta opción
  es algo engorrosa ya que debe existir un archivo XML por recurso, es
  decir, no puedes agrupar varios registros en un mismo fichero XML. Los
  socios son responsables del servidor SFTP. Se admiten modos de
  autentificación.
\item
  \textbf{FTP(S)}: es otro protocolo de transferencia de ficheros.
  Presenta las mismas características de importación que SFTP.
\item
  \textbf{Workspace}: se pueden subir directamente los registros en el
  \emph{workspace} de D4Sciente (ARIADNEplus Gateway). Cada socio tiene
  su propia carpeta donde puede ir almacenando los documentos XML
  (metadatos) que desee importar.
\end{enumerate}

Dado que este sistema es inaccesible para la mayoría de los miembros
(incluido el CENIEH), se debe escoger una de esas opciones y
comunicársela al coordinador responsable. Una vez realizada la
importación, se deben facilitar tres datos:

\begin{itemize}
\tightlist
\item
  Qué ficheros (\emph{.xml}) de los importados se desean publicar.
\item
  Cuál es el identificador del fichero de definición de mapeo (e.g.
  \emph{Mapping/621}) que transformará el esquema de metadatos presente
  en tus ficheros al esquema AO-Cat.
\item
  Opcionalmente, el enlace a tu colección de periodO y/o el fichero de
  mapeo (\emph{.json}) del vocabulario.
\end{itemize}

Los conjuntos de datos del CENIEH están almacenados de forma local,
exceptuando el CIR. Por ello, de entre todas las opciones posibles, la
única forma válida de importar metadatos sería a través del
\emph{Workspace}.

\hypertarget{simulaciuxf3n-de-publicaciuxf3n}{%
\subsubsection{Simulación de
publicación}\label{simulaciuxf3n-de-publicaciuxf3n}}

Una vez establecida la comunicación con el coordinador responsable del
proceso de importación, se debe esperar a su respuesta. Dependiendo del
contenido de la respuesta, se pueden tomar dos caminos:

\begin{enumerate}
\def\labelenumi{\arabic{enumi}.}
\tightlist
\item
  Nos indican que todas las partes del proceso (metadatos, mapeo,
  enriquecimiento) son correctos. En tal caso, los metadatos propuestos
  estarían ya disponibles desde el portal fantasma de ARIADNEplus. Este
  es idéntico al original con la única diferencia de que sólo tienen
  acceso los miembros del proyecto.
\item
  Nos indican que alguna parte del proceso no es correcta. Ante esta
  situación, se debe volver hacia atrás en el proceso de integración
  para solventar los conflictos señalados por el coordinador.
\end{enumerate}

\hypertarget{publicaciuxf3n}{%
\subsubsection{Publicación}\label{publicaciuxf3n}}

Si en la fase previa se ha obtenido una respuesta satisfactoria, el
miembro que inició el proceso de integración sería ya capaz de observar
el resultado final. A continuación, deberá comunicarse de nuevo con el
responsable de la importación para indicarle sus impresiones. Se pueden
dar dos situaciones:

\begin{enumerate}
\def\labelenumi{\arabic{enumi}.}
\tightlist
\item
  El resultado es favorable. Ante esta situación el coordinador lleva a
  cabo la publicación de los datos en el portal original.
\item
  No se esperaba el resultado obtenido. En tal caso, se deben mantener
  las conversaciones hasta llegar a una solución.
\end{enumerate}

En el caso de que todo haya salido según lo planeado, el proceso de
integración para los conjuntos de datos publicados quedaría suspendido.
Existe la posibilidad de reactivar este proceso en el caso de que se
deseen actualizar ciertos datos, sin embargo, hay que tener en cuenta
que cualquier cambio en la estructura de los datos supondría tener que
volver a realizar el proceso desde 0.

\hypertarget{desarrollo}{%
\section{Desarrollo}\label{desarrollo}}

Recordemos que en la fase anterior se anotaron todos los aspectos
relevantes del proceso de integración, incluyendo además los problemas
de incompatibilidad encontrados entre dicho proceso y los datos
propuestos por el CENIEH. Es en esta fase cuando se aplican las
competencias y los conocimientos adquiridos a lo largo del grado con el
objetivo de desarrollar una infraestructura \emph{software} que sea
capaz de guiar a los operarios del CENIEH en el proceso de integración
y, además, resuelva los problemas mencionados en la fase anterior.

\hypertarget{omeka-como-aplicaciuxf3n-principal}{%
\subsection{\texorpdfstring{\emph{Omeka} como aplicación
principal}{Omeka como aplicación principal}}\label{omeka-como-aplicaciuxf3n-principal}}

Desarrollar desde cero una infraestructura \emph{software} que cumpliera
con todos los requisitos propuestos no era viable debido a la limitación
temporal del proyecto. Por este motivo, se decidió utilizar
\emph{software} de terceros que cumpliera con un mínimo de
\textbf{requisitos}:

\begin{itemize}
\tightlist
\item
  Permitir la \textbf{gestión de metadatos}: los archivos de información
  involucrados son metadatos, por tanto, se necesita un sistema que
  permita realizar todo tipo de tareas de gestión sobre este tipo de
  datos.
\item
  Disponer de \textbf{herramientas de importación y exportación}: los
  datos de origen necesitarán ser importados a la plataforma para
  realizar sobre ellos las operaciones oportunas. Una vez gestionados,
  deberán ser exportados para someterlos al proceso de integración.
\item
  Ser \textbf{software libre}: este requisito era fundamental ya que,
  para poder adaptar la infraestructura a las necesidades del proyecto,
  se debe tener total libertad a la hora de ejecutar, copiar,
  distribuir, estudiar, modificar y mejorar el \emph{software}.
\end{itemize}

Se consideraron varios productos \emph{software} para acabar escogiendo
\href{https://omeka.org/classic/}{Omeka Classic}. Una de las
características que hacen de la aplicación una magnífica plataforma para
el proyecto es su \textbf{escalabilidad}. Gracias a su sistema de
\textbf{complementos} o \emph{plugins}, cualquier programador tiene la
posibilidad de adaptarla a sus necesidades individuales sin necesidad de
modificar el código base de la aplicación.

Actualmente, \emph{Omeka} cuenta con una gran cantidad de \emph{plugins}
disponibles tanto en su \href{https://omeka.org/classic/plugins/}{página
oficial} como en
\href{https://daniel-km.github.io/UpgradeToOmekaS/omeka_plugins.html}{Github}.
Esto es posible gracias a la extensa comunidad de usuarios que le
respalda. Parte de esos \emph{plugins} se han podido utilizar para
adaptar la infraestructura a las necesidades del proyecto, sin embargo,
se han tenido que desarrollar nuevos \emph{plugins} para cubrir
requisitos específicos. Además, se han llevado a cabo modificaciones
sobre alguno de los \emph{plugins} de terceros utilizados.

Por tanto, parte de las tareas de esta fase están relacionadas con la
creación y modificación de \emph{plugins} para \emph{Omeka}.

\hypertarget{adaptaciuxf3n-de-la-plataforma}{%
\subsubsection{Adaptación de la
plataforma}\label{adaptaciuxf3n-de-la-plataforma}}

Los complementos o \emph{plugins} son capaces de añadir nuevas
funcionalidades a \emph{Omeka} gracias a que esta tiene implementado un
sistema de ganchos o \emph{hooks}. Estos nos permiten acoplar código en
puntos específicos del flujo de ejecución de la aplicación, evitando así
tener que alterar el código base de esta.

Dentro de la aplicación se pueden encontrar dos tipos distintos de
\emph{hooks}: \emph{hooks} de acción y filtros (\emph{filters}).

\hypertarget{hooks-de-acciuxf3n}{%
\paragraph{\texorpdfstring{\emph{Hooks} de
acción}{Hooks de acción}}\label{hooks-de-acciuxf3n}}

Este tipo de \emph{hook} permite añadir la ejecución de funciones en
puntos de ejecución específicos.

Por ejemplo, en el caso de que se quiera introducir un formulario en una
página de \emph{Omeka}, se debería utilizar el \emph{action hook}
alojado en dicha página para ejecutar la función encargada de imprimir
el código HTML del formulario. En este ejemplo, la función no retornaría
nada ya que se limita a imprimir código, y es que en este tipo de
\emph{hooks} la función no tiene por qué devolver nada.

En los archivos de Omeka se pueden localizar estos \emph{hooks} buscando
la función \emph{fire\_plugin\_hook()}. Una vez encontrada, desde el
\emph{plugin} que estamos desarrollando, haciendo uso de la interfaz
\emph{Omeka\_Plugin\_AbstractPlugin}, bastaría con añadir este
\emph{hook} a la lista \emph{\_hooks} e instanciar el método
correspondiente, el cual siempre tiene la nomenclatura
\emph{hook\textless NombreDelHook\textgreater()}.

\begin{figure}
\hypertarget{actionhooks}{%
\centering
\includegraphics{../_static/images/actionhooks.png}
\caption{Ejemplo de hook de acción}\label{actionhooks}
}
\end{figure}

En el ejemplo vemos como \emph{fire\_plugin\_hook()} tiene dos
parámetros de entrada, el primero indica el nombre del \emph{hook} y el
segundo almacena los argumentos de entrada que tendrá la función que
almacena la acción.

\hypertarget{filtros-filters}{%
\paragraph{\texorpdfstring{Filtros
(\emph{Filters})}{Filtros (Filters)}}\label{filtros-filters}}

Los filtros permiten, al igual que los \emph{hooks} de acción, ejecutar
funciones propias en puntos específicos de la aplicación. Sin embargo,
el objetivo de estos es algo distinto ya que no pretenden modificar
código sino alterar los datos de una determinada variable.

Las funciones implicadas deben tener un parámetro de entrada y otro de
salida de forma que, desde el interior de la función, se procesa el
valor de entrada y se devuelve el valor resultante.

En los archivos de \emph{Omeka} se pueden localizar estos \emph{hooks}
buscando la función \emph{apply\_filters()}. Una vez encontrada, existen
dos formas de usar ese filtro:

\begin{enumerate}
\def\labelenumi{\arabic{enumi}.}
\tightlist
\item
  Utilizando la interfaz \emph{Omeka\_Plugin\_AbstractPlugin} es posible
  utilizar el filtro añadiendo su nombre a la lista \emph{\_filters}. A
  continuación, se añadiría el método público con el nombre
  \emph{filter} seguido del nombre del filtro.
\end{enumerate}

\begin{figure}
\hypertarget{filterhooksA}{%
\centering
\includegraphics{../_static/images/filterhooksA.png}
\caption{Ejemplo de \emph{filter hook}}\label{filterhooksA}
}
\end{figure}

\begin{enumerate}
\def\labelenumi{\arabic{enumi}.}
\setcounter{enumi}{1}
\tightlist
\item
  Utilizando el método \emph{add\_filter()}, se puede utilizar el filtro
  pasando como primer parámetro el nombre del filtro implicado y como
  segundo parámetro la función que se ejecutará. En este caso, el nombre
  de la función es personalizable. Además, se puede pasar un tercer
  parámetro para indicar la prioridad de nuestro \emph{hook}, es decir,
  si existiera más de un \emph{plugin} utilizando ese mismo filtro, se
  ejecutaría la función de cada uno en función de su prioridad, de mayor
  a menor prioridad. Por defecto, todos los \emph{filtros} de cada
  \emph{plugin} tienen una prioridad de 10, por lo que el orden de
  ejecución se determina por la fecha de instalación, de más antiguos a
  más nuevos.
\end{enumerate}

\begin{figure}
\hypertarget{filterhooksB}{%
\centering
\includegraphics{../_static/images/filterhooksB.png}
\caption{Segundo ejemplo de \emph{filter hook}}\label{filterhooksB}
}
\end{figure}

\hypertarget{entornos-de-trabajo}{%
\subsection{Entornos de trabajo}\label{entornos-de-trabajo}}

Durante la fase de desarrollo, se ha trabajado sobre dos entornos:

\begin{itemize}
\tightlist
\item
  \textbf{Entorno de desarrollo}: se actualiza al cometer cambios sobre
  la rama \emph{develop}. Permite llevar un seguimiento diario del
  estado de la aplicación durante el desarrollo de la misma. Es público.
\item
  \textbf{Entorno de producción}: se actualiza al cometer cambios sobre
  la rama \emph{main}. En su interior se puede encontrar una versión
  estable de la aplicación. El intervalo de tiempo de actualización gira
  entorno a las dos semanas. Es privado, sólo tienen acceso los miembros
  del CENIEH.
\end{itemize}

\hypertarget{despliegue-de-la-infraestructura}{%
\subsection{Despliegue de la
infraestructura}\label{despliegue-de-la-infraestructura}}

Para llevar a cabo el despliegue de la infraestructura se ha utilizado
la herramienta \emph{Github Actions}. Dependiendo del entorno de
trabajo, se ha procedido de una manera u otra:

\hypertarget{servidor-de-desarrollo}{%
\subsubsection{Servidor de desarrollo}\label{servidor-de-desarrollo}}

A través de la herramienta \emph{Github Actions} se ha automatizado el
despliegue de la infraestructura sobre el servidor de desarrollo. A esta
técnica se la conoce como despliegue continuo.

\begin{figure}
\hypertarget{cicd}{%
\centering
\includegraphics{../_static/images/cicd.png}
\caption{Despliegue continuo de la aplicación}\label{cicd}
}
\end{figure}

En la \texttt{cicd} he representado el proceso mediante el cual se lleva
a cabo el despliegue. Con el \emph{workflow} configurado y alojado en la
ruta \emph{.github/workflows} de mi repositorio en GitHub, cuando
ejecuto un \emph{push} sobre la rama \emph{develop}, si los cambios
cometidos afectan a cualquier carpeta que no sea la de \emph{/docs}, se
ejecutan las acciones correspondientes al despliegue de mi aplicación,
las cuales se pueden apreciar en la imagen.

\hypertarget{servidor-de-producciuxf3n}{%
\subsubsection{Servidor de producción}\label{servidor-de-producciuxf3n}}

Sobre el servidor de producción no se ha podido automatizar el
despliegue debido a que el acceso a este era privado, es decir, no se
podía establecer comunicación desde el exterior sin previa conexión al
VPN del CENIEH y el posterior acceso vía \emph{ssh} al servidor.

\begin{figure}
\hypertarget{dockerdeploy}{%
\centering
\includegraphics{../_static/images/dockerdeploy.png}
\caption{Despliegue "semi-continuo" de la
aplicación}\label{dockerdeploy}
}
\end{figure}

Como solución a este inconveniente, se automatizó por separado la
compilación y publicación de la imagen \emph{Docker} asociada a nuestra
aplicación. De esta manera, cada vez que se cometía un cambio sobre la
rama \emph{main}, se ejecutaba dicho proceso, actualizando la imagen
publicada en el repositorio de \emph{DockerHub}. Finalizado el proceso,
se accedía al servidor de producción y se desplegaba manualmente la
infraestructura. Durante el despliegue, se recogían las imágenes desde
\emph{DockerHub}, incluyendo la imagen actualizada de nuestra
aplicación.

\end{document}

% Options for packages loaded elsewhere
\PassOptionsToPackage{unicode}{hyperref}
\PassOptionsToPackage{hyphens}{url}
%
\documentclass[
]{article}
\usepackage{lmodern}
\usepackage{amssymb,amsmath}
\usepackage{ifxetex,ifluatex}
\ifnum 0\ifxetex 1\fi\ifluatex 1\fi=0 % if pdftex
  \usepackage[T1]{fontenc}
  \usepackage[utf8]{inputenc}
  \usepackage{textcomp} % provide euro and other symbols
\else % if luatex or xetex
  \usepackage{unicode-math}
  \defaultfontfeatures{Scale=MatchLowercase}
  \defaultfontfeatures[\rmfamily]{Ligatures=TeX,Scale=1}
\fi
% Use upquote if available, for straight quotes in verbatim environments
\IfFileExists{upquote.sty}{\usepackage{upquote}}{}
\IfFileExists{microtype.sty}{% use microtype if available
  \usepackage[]{microtype}
  \UseMicrotypeSet[protrusion]{basicmath} % disable protrusion for tt fonts
}{}
\makeatletter
\@ifundefined{KOMAClassName}{% if non-KOMA class
  \IfFileExists{parskip.sty}{%
    \usepackage{parskip}
  }{% else
    \setlength{\parindent}{0pt}
    \setlength{\parskip}{6pt plus 2pt minus 1pt}}
}{% if KOMA class
  \KOMAoptions{parskip=half}}
\makeatother
\usepackage{xcolor}
\IfFileExists{xurl.sty}{\usepackage{xurl}}{} % add URL line breaks if available
\IfFileExists{bookmark.sty}{\usepackage{bookmark}}{\usepackage{hyperref}}
\hypersetup{
  pdftitle={Trabajos relacionados},
  hidelinks,
  pdfcreator={LaTeX via pandoc}}
\urlstyle{same} % disable monospaced font for URLs
\usepackage{longtable,booktabs}
% Correct order of tables after \paragraph or \subparagraph
\usepackage{etoolbox}
\makeatletter
\patchcmd\longtable{\par}{\if@noskipsec\mbox{}\fi\par}{}{}
\makeatother
% Allow footnotes in longtable head/foot
\IfFileExists{footnotehyper.sty}{\usepackage{footnotehyper}}{\usepackage{footnote}}
\makesavenoteenv{longtable}
\setlength{\emergencystretch}{3em} % prevent overfull lines
\providecommand{\tightlist}{%
  \setlength{\itemsep}{0pt}\setlength{\parskip}{0pt}}
\setcounter{secnumdepth}{-\maxdimen} % remove section numbering

\title{Trabajos relacionados}
\author{}
\date{}

\begin{document}
\maketitle

Algunos socios del proyecto ARIADNEplus han adoptado una solución muy
similar a la propuesta en el presente proyecto, es decir, han hecho uso
de aplicaciones \emph{software} de terceros para la gestión de sus
(meta)datos y las han adaptado según sus necesidades. A continuación, se
muestran aquellos casos que guardan una mayor relación con el proyecto.

\hypertarget{casos-similares}{%
\section{Casos similares}\label{casos-similares}}

\hypertarget{fasti-online}{%
\subsection{Fasti Online}\label{fasti-online}}

Fasti Online\footnote{"Fasti Online." \url{http://www.fastionline.org/}}
es un proyecto liderado por la Asociación Internacional de Arqueología
Clásica (AIAC)\footnote{"AIAC -- Associazione Internazionale di
  Archeologia Classica." \url{http://www.aiac.org/}} y el \emph{Center
for the Study of Ancient Italy} (CSAI)\footnote{"CSAI -- Center for the
  Study of Ancient Italy." \url{http://csaitx.org/}} de la Universidad
de Texas, Austin\footnote{"University of Texas at Austin."
  \url{https://www.utexas.edu/}}. Su principal objetivo es proporcionar
una infraestructura \emph{software} que permita almacenar, gestionar y
publicar registros relacionados con la arqueología.

Para tal fin, han utilizado como base la aplicación \emph{software}
denominada \href{https://ark.lparchaeology.com/}{ARK}. Esta es una
aplicación web que provee servicios como la gestión, compartición y
transformación (mapeo) de (meta)datos. Además, la aplicación es de
código abierto, lo que significa que es personalizable y extensible.

La incorporación de \emph{Fasti Online} al proyecto \emph{ARIADNE} y,
posteriormente, al proyecto \emph{ARIADNEplus}, ha impulsado la
implementación de nuevas funcionalidades sobre la aplicación \emph{ARK}
como, por ejemplo, la integración de datos espaciales, nuevos mecanismos
de búsqueda y otros servicios web como, por ejemplo, el protocolo
\emph{OAI-PMH}.

\hypertarget{conicet}{%
\subsection{CONICET}\label{conicet}}

El Consejo Nacional de Investigaciones Científicas y Técnicas
(CONICET)\footnote{"CONICET -- Consejo Nacional de Investigaciones
  Científicas y Técnicas." \url{https://www.conicet.gov.ar/}} es el
principal organismo dedicado a la promoción de la ciencia y la
tecnología en Argentina. Este, al igual que el CENIEH, es una de las
nuevas incorporaciones al proyecto \emph{ARIADNEplus} y, como tal, han
tenido que adaptarse para satisfacer los requisitos del proyecto.

La solución planteada por este organismo es muy similar a la del
presente proyecto. Están desarrollado una infraestructura
\emph{software} que permita a los operarios del CONICET gestionar y
publicar sus conjuntos de datos adoptando un esquema de metadatos
compatible con \emph{ARIADNEPlus}. La aplicación \emph{software} que han
decidido adaptar ha sido \href{https://duraspace.org/dspace/}{Dspace
5.5} . Se puede acceder a su infraestructura desde el siguiente
\href{https://suquia.ffyh.unc.edu.ar/}{enlace} .

\hypertarget{dans}{%
\subsection{DANS}\label{dans}}

DANS (\emph{Data Archiving and Networked Services})\footnote{"DANS --
  Data Archiving and Networked Services." \url{https://dans.knaw.nl/en}}
es una institución de los Paises Bajos cuya misión principal es
proporcionar las herramientas necesarias a investigadores para hacer que
sus datos sean accesibles, interoperables y reutilizables.

La organización DANS es responsable del desarrollo y mantenimiento del
repositorio digital \href{https://dendro.dans.knaw.nl/}{DCCD}, el cual
es considerado como la principal red de (meta)datos
arqueológicos/históricos existente en Europa. Entró en funcionamiento en
2011. Dentro del \emph{DCCD}, laboratorios belgas, daneses, holandeses,
alemanes, letones, polacos y españoles publican contenido fruto de la
investigación de, entre otros: sitios arqueológicos (incluidos paisajes
antiguos), construcciones, pinturas, esculturas e instrumentos
musicales.

Esta organización participó en el proyecto \emph{ARIADNE} y,
actualmente, forma parte del proyecto \emph{ARIADNEplus}. Con el
objetivo de mejorar la integración europea de datos dendrocronológicos
ofrecen, de forma gratuita, la misma solución \emph{software} empleada
en su proyecto \emph{DCCD}, la cual es compatible con el proyecto
\emph{ARIADNE}. Está disponible en
\href{https://github.com/DANS-KNAW/dccd-webui}{Github} .

\hypertarget{comparativa-entre-soluciones-software}{%
\section{\texorpdfstring{Comparativa entre soluciones
\emph{software}}{Comparativa entre soluciones software}}\label{comparativa-entre-soluciones-software}}

\begin{longtable}[]{@{}lllll@{}}
\caption{Comparativa de las principales características de las
aplicaciones \emph{software} escogidas por cada socio.}\tabularnewline
\toprule
Caraterísticas & Omeka Classic (CENIEH) & ARK (Fasti Online) & DSpace
(CONICET) & DCCD (DANS)\tabularnewline
\midrule
\endfirsthead
\toprule
Caraterísticas & Omeka Classic (CENIEH) & ARK (Fasti Online) & DSpace
(CONICET) & DCCD (DANS)\tabularnewline
\midrule
\endhead
Tipo de aplicación & Web & Web & Web & Web\tabularnewline
Lenguaje de programación principal & PHP & PHP & Java &
Java\tabularnewline
Gestión de metadatos & ✔ & ✔ & ✔ & ✔\tabularnewline
Importación masiva de metadatos & ✔ & ✔ & ✔ & ✔\tabularnewline
Exportación masiva de metadatos & ✔ & ✔ & ✔ & ✔\tabularnewline
Edición masiva de metadatos & ✔ & ✘ & ✘ & ✘\tabularnewline
Cobertura espacial & ✔ & ✔ & ✔ & ✔\tabularnewline
Cobertura temporal & ✘ & ✔ & ✘ & ✔\tabularnewline
Protocolo OAI-PMH & ✔ & ✔ & ✔ & ✘\tabularnewline
Herramientas de apoyo en la integración con ARIADNEplus & ✔ & ✘ & ✘ &
✘\tabularnewline
Herramientas para la transformación de metadatos & ✔ & ✔ & ✘ &
✘\tabularnewline
Sistema de usuarios & ✔ & ✔ & ✔ & ✔\tabularnewline
Almacenamiento de ficheros & ✔ & ✔ & ✔ & ✘\tabularnewline
Asistencia técnica gratuita & ✔ & ✘ & ✘ & ✘\tabularnewline
Interfaz pública & ✔ & ✔ & ✔ & ✔\tabularnewline
Interfaz intuitiva & ✔ & ✘ & ✔ & ✘\tabularnewline
Sistema de \emph{plugins} & ✔ & ✘ & ✔ (*) & ✘\tabularnewline
Sistema de plantillas & ✔ & ✘ & ✘ & ✘\tabularnewline
Comunidad de usuarios activa & ✔ & ✘ & ✔ & ✘\tabularnewline
Manuales de documentación detallados & ✔ & ✘ & ✘ & ✘\tabularnewline
Última actualización & 2020 & 2018 & 2020 & 2015\tabularnewline
\bottomrule
\end{longtable}

\emph{(*) Servicio de pago.}

Basándonos en el contenido de la \texttt{compsolsoft}, se listarán los
puntos fuertes y débiles que presenta la aplicación del proyecto frente
a las propuestas de los otros socios.

\hypertarget{puntos-fuertes}{%
\subsection{Puntos fuertes}\label{puntos-fuertes}}

\begin{itemize}
\tightlist
\item
  Gran parte de la configuración de la aplicación puede realizarse desde
  la interfaz gráfica, sin necesidad de modificar ficheros internos que
  requieran un mínimo de conocimiento de la estructura interna de la
  aplicación, como pasa en aplicaciones como \emph{ARK} o \emph{DCCD}.
  Esto facilita en gran medida las labores de configuración de la
  aplicación.
\item
  Al requerir una infraestructura \emph{LAMP} para su despliegue, la
  instalación de la aplicación es relativamente sencilla en comparación
  con las otras aplicaciones. Además, gracias al presente proyecto, es
  posible instalar la aplicación a través de tecnologías como
  \emph{Docker} o \emph{Kubernetes}, facilitando aún más su despliegue.
\item
  De entre todas las soluciones mostradas es, sin duda, la más sencilla
  y segura de adaptar y personalizar. Esto es gracias al sistema de
  complementos (\emph{plugins}) y plantillas (\emph{themes}) que
  incorpora.
\item
  Gracias a las labores de desarrollo llevadas a cabo en el presente
  proyecto, dispone de herramientas de apoyo para la integración de
  conjuntos de datos en ARIADNEplus.
\item
  La comunidad de usuarios con la que cuenta \emph{Omeka Classic} es
  superior a la de sus competidores. Muchos usuarios comparten sus
  propios desarrollos, tanto complementos como plantillas, de forma que
  estos pueden ser reutilizados o incluso mejorados por otros usuarios.
  Además, existe un foro desde donde los expertos de \emph{Omeka},
  incluídos los líderes del proyecto, brindan soporte técnico gratuito a
  otros usuarios de la aplicación.
\item
  La documentación disponible es, tanto para usuarios como para
  desarrolladores, la más clara y detallada de todas las aplicaciones
  mostradas.
\item
  Actualmente el proyecto \emph{Omeka} continúa en desarrollo, es decir,
  siguen saliendo nuevas actualizaciones con mejoras y funcionalidades
  nuevas para la aplicación. Sin embargo, otros proyectos como
  \emph{ARK} o \emph{DCCD} están obsoletos.
\end{itemize}

\hypertarget{puntos-duxe9biles}{%
\subsection{Puntos débiles}\label{puntos-duxe9biles}}

\begin{itemize}
\tightlist
\item
  Actualmente, no dispone de ningún mecanismo que identifique aquellos
  (meta)datos cuyo contenido sea un periodo temporal (e.g. "1190 BCE") y
  los procese de tal forma que estos sean mostrados dentro de una línea
  temporal y a su vez puedan ser un criterio aislado de búsqueda.
\item
  No posee las ventajas que proporciona el lenguaje de programación
  \emph{Java} utilizado tanto en \emph{DSpace} como en \emph{DCCD}. Este
  es más rápido y presenta un mejor rendimiento al ser un lenguaje
  compilado. Además, posee una estructura más ordenada y es mucho más
  seguro que PHP.
\end{itemize}

\end{document}

\capitulo{7}{Conclusiones y Líneas de trabajo futuras}

En este último apartado, se exponen las conclusiones extraídas tras un
breve análisis objetivo del trabajo realizado. Además, se proponen
nuevas perspectivas para las posibles líneas de trabajo futuras.

\section{Conclusiones}

A continuación se listan las conclusiones más relevantes que se han
obtenido tras finalizar el proyecto.

\begin{itemize}
\tightlist
\item
  En cuanto a los objetivos generales del proyecto, considero que se han
  cumplido ambos. Los operarios del \emph{CENIEH} cuentan con una
  aplicación que les facilita el proceso de integración de sus datos en
  \emph{ARIADNEplus} y, además, se ha conseguido integrar una de las
  colecciones propuestas.
\item
  Durante la fase de investigación del proyecto, se han aprendido
  multitud de técnicas y conocimientos nuevos relacionados con la
  creación, implementación y gestión de metadatos.
\item
  El ser parte de un proyecto internacional como \emph{ARIADNEplus} nos
  ha permitido conocer nuevos métodos de trabajo como, por ejemplo, la
  utilización de entornos de investigación virtuales (\emph{VREs}).
  Gracias a estos hemos podido comunicarnos con los demás socios del
  proyecto, utilizar servicios y herramientas comunes, y compartir
  recursos digitales de todo tipo.
\item
  En la parte de desarrollo del proyecto se han aplicado la mayoría de
  los conocimientos adquiridos durante el grado. Asímismo, se han
  utilizado otras materias que han requerido un estudio especial como
  \emph{PHP}, \emph{Zend Framework}, \emph{Hooking}, etc.
\item
  El desarrollo de \emph{plugins} o complementos para la adaptación de
  la aplicación propuesta ha supuesto una experiencia totalmente nueva
  que me ha permitido conocer cómo funcionan este tipo de aplicaciones.
\item
  En el proyecto se han aplicado técnicas de integración continua que
  han permitido agilizar muchas de las tareas involucradas en su
  desarrollo, afectando positivamente a la calidad del código y a la
  depuración de errores.
\end{itemize}

\section{Líneas de trabajo futuras}

Se pueden tomar dos caminos distintos para mejorar la aplicación
propuesta:

\begin{enumerate}
\def\labelenumi{\arabic{enumi}.}
\tightlist
\item
  Desarrollar nuevos complementos (\emph{plugins}) que añadan nuevas
  funcionalidades.
\item
  Extender la funcionalidad de los complementos propuestos en este
  proyecto.
\end{enumerate}

A continuación se exponen las funcionalidades que pueden resultar
interesantes añadir en la plataforma.

\begin{itemize}
\tightlist
\item
  Dar algún tipo de soporte a los periodos temporales que pudieran
  aparecer dentro de los (meta)datos. Por ejemplo, representarlos
  gráficamente dentro de una linea temporal.
\item
  Sugerir periodos temporales del cliente \emph{PeriodO} a la hora de
  rellenar el metadato "Temporal Coverage". De esta manera, se podrá
  enriquecer dicho metadato en la fase de integración correspondiente.
\end{itemize}

En cuanto a las posibles mejoras de los complementos:

\begin{itemize}
\item
  Complemento \emph{ARIADNEplus Tracking}: introducir nuevas funciones
  en alguna de las fases de los \emph{tickets}.

  \begin{quote}
  \begin{itemize}
  \tightlist
  \item
    \emph{Fase 1}: Poder editar los ítems desde la misma ventana, sin
    necesidad de desplazarse al gestor de ítems.
  \item
    \emph{Fase 3}: Previsualizar la colección de \emph{periodO} indicada
    por el usuario y poder adjuntar el fichero de definición de mapeo
    desde la misma ventana.
  \item
    \emph{Fase 4}: Previsualizar los ítems publicados en el portal
    fantasma de \emph{ARIADNEplus} a partir del enlace \emph{SPARQL}.
  \end{itemize}
  \end{quote}
\item
  Complemento \emph{Geolocation}: introducir localizaciones con áreas
  poligonales (hasta ahora solo se pueden simples o rectangulares).
\item
  Complemento \emph{Bulk Metadata Editor}: introducir nuevas acciones de
  edición como, por ejemplo, poder asignar más de dos valores a un mismo
  metadato.
\item
  Complemento \emph{OAI-PMH Harvester}: poder programar recolecciones de
  metadatos en determinados intervalos de tiempo.
\item
  Complemento \emph{AutoDublinCore}: \item
  dado que la localización de los datos es imprescindible, crear un
  sistema que en caso de que el metadato que se encarga de ello
  ("\emph{Spatial Coverage}) se encuentre vacío, busque en el contenido
  de los demás metadatos (e.g. \emph{Title}, \emph{Description}, etc.)
  una localización y, en caso de encontrarla, actualizar el contenido
  del metadato con dicha localización.
\item 
  Traducir todos los complementos desarrollados en este proyecto a otros idiomas.
\end{itemize}



\bibliographystyle{plain}
\bibliography{bibliografia}

\begin{mdframed}[outerlinecolor=black,outerlinewidth=2pt,linecolor=cccolor,middlelinewidth=3pt,roundcorner=10pt]
  This work is licensed under a Creative Commons Attribution 4.0 International License.
  \begin{center}
    \includegraphics[scale=1]{by.png}
  \end{center}
\end{mdframed}

\end{document}