\capitulo{3}{Conceptos teóricos}

A lo largo de este apartado se van a exponer los conceptos teóricos
relacionados con las dos primeras fases en las que se divide el
proyecto, que son investigación y desarrollo.

\section{Conceptos teóricos relativos a la investigación}

En esta sección se definen todos aquellos conceptos relacionados con la
investigación previa al desarrollo e implementación de la
infraestructura \emph{software} propuesta.

\subsection{\emph{ARIADNEplus}}

\emph{ARIADNEplus} \cite{arip:web} es la continuación del
proyecto \emph{ARIADNE} \cite{ari:web}, el cual fue fundado
por la Comisión Europea en febrero de 2013. Nació con el propósito de
estimular la investigación en áreas relacionadas con la arqueología
mediante la integración de diversas infraestructuras de datos
arqueológicas situadas en Europa. Fruto de este proyecto surgió un
catálogo \emph{on-line} de (meta)datos referentes a conjuntos de datos
que incluían reportes no publicados, imágenes, mapas, bases de datos, y
otros tipos de información arqueológica.

Este segundo proyecto forma parte del programa \emph{H2020}, fundado
también por la Comisión Europea. El proyecto se encuentra en desarrollo
desde enero de 2019 y tiene previsto una duración total de 48 meses. A
través de \emph{ARIADNEplus}, se actualizarán y extenderán los datos del
catálogo \emph{on-line} anterior añadiendo a los mismos dimensión
geográfica y temporal. 

Además, se van a incorporar más organizaciones
arqueológicas Europeas (entre ellas el CENIEH). También proveerá nuevos
servicios en la nube para procesar y re-utilizar los datos incluidos en
su portal.

\subsection{\emph{CENIEH}}

El Centro Nacional de Investigación sobre la Evolución
Humana \cite{cenieh:web}, tambien conocido como
CENIEH, es una Infraestructura Científica y Técnica Singular (ICTS)
abierta al uso de la comunidad científica y tecnológica, en la que se
desarrollan investigaciones en el ámbito de la evolución humana durante
el Neógeno superior y Cuaternario, promoviendo la sensibilización y
transferencia de conocimientos a la sociedad e impulsando y apoyando la
realización y colaboración en excavaciones de yacimientos de estos
periodos, tanto españoles como de otros países.

Además, el CENIEH es responsable de la conservación, restauración,
gestión y registro de las colecciones paleontológicas y arqueológicas
procedentes de las excavaciones de Atapuerca y otros yacimientos tanto
nacionales como internacionales de similares características.

\subsection{\emph{CIR}}

El CIR \cite{cir:web} (CENIEH Institutional Repository) es el
repositorio bibliográfico institucional del CENIEH. Alberga toda la
información fruto de la actividad investigadora desarrollada en el
CENIEH como, por ejemplo, publicaciones científicas. 

Toda la información que recoge está organizada en ítems que pertenecen 
a una colección, que a su vez forman parte de una comunidad. Cada ítem 
tiene asignado un conjunto de metadatos que describen al objeto digital 
que contiene. El esquema de metadatos utilizado por la plataforma se le 
conoce como \emph{Dublin Core}.

\subsection{Metadatos}

Los metadatos proporcionan la información mínima necesaria para
identificar un recurso, pudiendo incluir información descriptiva sobre
el contexto, calidad y condición o característica del dato \cite{art:meta}. Puede resultar
algo complejo de entender ya que podemos reducir su definición a ``son
datos que describen otros datos''.

Para aportar algo de claridad a esta definición se aplicará el concepto de
``metadato'' tomando como ejemplo una biblioteca. En este contexto, el
conjunto de datos estaría formado por los libros y el conjunto de
metadatos se correspondería con las fichas asociadas a cada libro. Este
ejemplo de metadato está algo anticuado ya que se presenta de una forma
física, no digital.

\imagen{ejemploMetadatos}{Ejemplo de metadatos.}{0.6}

En la actualidad, estas ``fichas'' se encuentran en formato digital a
través de lenguajes de marcado como \emph{XML} o \emph{RDF}.

\subsection{Esquema de metadatos}

Antes de introducir metadatos en cualquier catálogo, es necesario
indicar como van a estar organizados. Para llevar a cabo esta tarea hay
que definir antes un esquema de metadatos, también llamado modelo o
estándar.

Cada esquema está formado por un conjunto de campos de diferentes tipos,
los cuales siguen una estructura jerárquica en forma de árbol.

\imagen{diagramacampos}{Estructura básica de un esquema de metadatos.}{0.7}

En la Figura \ref{fig:diagramacampos}, se muestra la \textbf{estructura básica} 
de cualquier esquema:

\begin{quote}
\begin{itemize}
\item
  \textbf{Ontología}: es la raíz del esquema. Su función es agrupar los
  demás campos en una única unidad temática. Puede tener tres tipos de
  descendientes: Clase, Referencia o Metadato.
\item
  \textbf{Definición de Clase}: define una clase o subclase dentro de
  una ontología determinada, creando así una jerarquía de clases.

  \begin{quote}
  \begin{itemize}
  \tightlist
  \item
    \textbf{Atributo}: define un atributo para una determinada clase
    existente en la ontología.
  \end{itemize}
  \end{quote}
\item
  \textbf{Conjunto de Referencia}: define un conjunto de valores que
  pueden ser instanciados en el Atributo de una Clase o en el Metadato
  de un recurso.

  \begin{quote}
  \begin{itemize}
  \tightlist
  \item
    \textbf{Valor}: define el contenido de cada valor existente en un
    conjunto de referencia.
  \end{itemize}
  \end{quote}
\item
  \textbf{Metadato de Recurso}: define el metadato de un recurso
  determinado. Además, puede ser descendiente de otro metadato a modo de
  especificación.
\end{itemize}
\end{quote}

Cuando se define un atributo o un metadato, se debe indicar, además, el
tipo de contenido que va a adquirir, es decir, señalar qué se va a
introducir. Algunos pueden ser texto plano, otros coordenadas, fechas,
enlaces, etc.

\subsubsection{\emph{CIDOC-CRM}}

\emph{\textbf{CIDOC} \textbf{C}onceptual \textbf{R}eference \textbf{M}odel}
\cite{cidoc:web} (CRM) es una
ontología que ofrece definiciones y una estructura formal para describir
conceptos implícitos y explícitos, así como las relaciones utilizadas en
documentación sobre patrimonio cultural. 

\subsubsection{\emph{ACDM}}

El \emph{\textbf{A}RIADNE \textbf{C}atalogue \textbf{D}ata \textbf{M}odel} es
el modelo de datos utilizado por el catálogo antiguo de \emph{ARIADNE}. Sirve
para describir los recursos arqueológicos publicados por los
participantes del proyecto. El uso de \emph{ACDM} posibilita el descubrimiento,
acceso e integración de los citados recursos. Para formalizar este
modelo, se ha utilizado como base la ontología \emph{CIDOC CRM}, la cual se
adapta correctamente al dominio arqueológico.

\subsubsection{\emph{PEM}}

\emph{PEM} \cite{art:pem}(\emph{\textbf{P}ARTHENOS \textbf{E}ntities \textbf{M}odel}) es el esquema
de metadatos desarrollado en el proyecto \emph{PARTHENOS} \cite{parthenos:web} que extiende al
modelo \emph{CIDOC-CRM}. Está diseñado para ser lo suficientemente flexible
como para mapear los diferentes tipos de esquemas de metadatos
utilizados en todas las disciplinas académicas de manera uniforme.

\subsubsection{AO-Cat}

La ontología \emph{AO-Cat} \cite{art:aocat} (\emph{\textbf{A}RIADNE \textbf{O}ntology
\textbf{-} \textbf{Cat}alog}) deriva del modelo \emph{ACDM}, empleado
por el proyecto antiguo (\emph{ARIADNE}) para modelar recursos arqueológicos, y
del modelo \emph{PEM}, utilizado para modelar cualquier recurso gestionado por
una infraestructura de investigación.

Se podría decir que \emph{AO-Cat} es una
contracción del modelo \emph{ACDM} impulsada por la conceptualización
subyacente al \emph{PEM}. Además, \emph{AO-Cat} hereda del modelo \emph{PEM} su estrecha
relación con el modelo \emph{CIDOC-CRM}, el cual sirve para representar
cualquier aspecto relacionado con recursos arqueológicos.

\emph{AO-Cat} es el \textbf{modelo utilizado por el catálogo actual de ARIADNEplus} y,
por tanto, los metadatos de todos los socios del proyecto se tienen que transformar a este
modelo.

\subsection{Mapeo de datos (\emph{Data Mapping})}

El término ``mapeo'' puede utilizarse en múltiples contextos como, por
ejemplo, en la cartografía, matemáticas, neurociencia, etc. En esta
ocasión, se describirá el concepto relacionado con la informática, más
específicamente con la gestión de datos.

El mapeo de datos consiste en crear asignaciones entre dos elementos que
pertenecen a esquemas de datos distintos. En procesos como la
integración o migración de datos es fundamental llevar a cabo este tipo
de proceso debido a que, generalmente, el sistema al que se trasladan
los datos no utiliza la misma estructura que el sistema de partida.

\imagen{mapping}{Ejemplo de definición de mapeo entre el esquema ``Dublin Core'' y el modelo ``AO-Cat''.}{0.9}

\subsection{Enriquecimiento de datos (\emph{Data Enrichment})}

El enriquecimiento de datos es el proceso mediante el cual es posible
mejorar la calidad de los datos sin necesidad de procesarlos. Durante
este proceso, se fusionan los datos originales con datos de terceros
provenientes de una fuente autorizada externa. 

Para determinar la relación entre los datos originales y los externos se suele hacer uso de
herramientas auxiliares que permiten establecer dichas relaciones.

\imagen{enrichmentconcept}{Proceso de enriquecimiento de datos.}{0.5}

\subsection{\emph{D4Science} -- Entornos de investigación virtuales}

\emph{D4Science} \cite{dfour:web}
es una organización que ofrece una infraestructura de datos basada en
entornos de investigación virtuales (\emph{VREs} \cite{art:vre}). En este tipo de entornos el usuario cuenta con un espacio de trabajo virtual que le da la posibilidad de acceder a datos y compartir los suyos propios. Además, también cuenta con herramientas y capacidad de cómputo
para hacer uso de los datos en su proceso de investigación.


\subsubsection{\emph{ARIADNEplus Gateway}}

\emph{ARIADNEplus} cuenta con un portal en la plataforma
\emph{D4Science} denominado \emph{ARIADNEplus Gateway} \cite{aplusgat:web}. En él tiene
implementados varios entornos virtuales de investigación.
Cada uno de ellos ofrece una serie de servicios que facilitan el proceso
de integración a los miembros del proyecto. Actualmente, cuenta con tres
entornos virtuales, cada uno de los cuales tiene un fin específico:

\imagen{d4scienceVREs}{Entornos virtuales de investigación en D4Science.}{0.9}

\begin{itemize}
\tightlist
\item
  \emph{ARIADNEplus Aggregation Management}: entorno virtual donde
  los líderes del proyecto gestionan las importaciones de metadatos al
  catálogo. El acceso está restringido a los coordinadores del proyecto.
\item
  \emph{ARIADNEplus Mappings}: entorno virtual que da soporte a la
  conversión de metadatos (\emph{mapping}) para su integración en
  \emph{ARIADNEplus}.
\item
  \emph{ARIADNEplus Project}: entorno virtual que permite la
  colaboración y cooperación entre los beneficiarios del proyecto
  \emph{ARIADNEplus}.
\end{itemize}

\subsubsection{\emph{Workspace}}

Otro de los servicios que ofrece \emph{D4Science} es el
\emph{Workspace} \cite{dfourwork:web}. La idea principal de esta herramienta es que los
miembros de un determinado portal intercambien recursos digitales como,
por ejemplo, documentos, imágenes, vídeos, etc.

En este espacio de trabajo los miembros de \emph{ARIADNEplus} organizan y
comparten recursos relacionados con el proyecto como, por ejemplos,
guías, tutoriales, presentaciones, etc.

\imagen{workspace}{Espacio de trabajo (\emph{Workspace}) del proyecto \emph{ARIADNEPlus}.}{1}

Además, este mismo espacio se puede utilizar como medio de importación para el 
catálogo de \emph{ARIADNEPlus}.
Para tal fin, como podemos ver en el imagen, existen dos carpetas
públicas, \emph{Matched Vocabularies} y \emph{Metadata Ingestion}, en
cuyo interior se aloja una carpeta para cada miembro del proyecto. La misión de cada
carpeta es almacenar los ficheros de definición de mapeo de vocabulario (\emph{.json}) y los ficheros con los metadatos (\emph{.xml}). De esta manera, el coordinador puede acceder
a los datos necesarios para llevar a cabo la importación sin necesidad de usar medios externos.

\subsection{\emph{Getty AAT}}

\emph{Getty AAT} \cite{getty:web} es un vocabulario controlado y
estructurado que se emplea para describir elementos de arte,
arquitectura y material cultural. Está compuesto por términos generales
como, por ejemplo, ``Acueducto'', pero no contiene nombres propios como
``Acueducto de Segovia''. Actualmente cuenta con alrededor de 55.000
conceptos registrados, incluyendo 131.000 términos, descripciones,
citaciones bibliográficas, y otra información relacionada con las áreas
previamente mencionadas.

Además cuenta con una interfaz \emph{SPARQL} \cite{getty:sparql} que permite realizar consultas 
sobre los datos (\emph{RDF}) almacenados en su base de datos mediante el lenguaje \emph{SPARQL}.

\subsection{\emph{PeriodO}}

\emph{PeriodO} \cite{getty:web} es un
diccionario digital público donde se almacenan definiciones académicas
de periodos históricos, histórico-artísticos y arqueológicos. Este
proyecto es liderado por Adam Rabinowitz (Universidad de Texas, Austin)
y Ryan Shaw (Universidad de Carolina del norte, Chapel Hill).

\subsection{Tecnología \emph{GraphDB}}

\emph{ARIADNEplus} almacena todos los metadatos en un almacén de \emph{RDF}
(\emph{triplestore}) basado en la tecnología \emph{GraphDB} \cite{gdb:web}. 
Este tipo de tecnología utiliza \textbf{bases de datos orientadas a grafos}. Estas se
basan en un conjunto de objetos (vértices y aristas) que permiten
representar datos interconectados junto a las relaciones existentes
entre sí. 

Cada grafo está compuesto por nodos o vértices, que se
corresponden con los datos (objetos), y aristas o arcos, que serían las
relaciones entre los datos. 

La estructura de este tipo de bases de datos
puede adoptar dos formas: \emph{Labeled-Property Graph} (grafo de
propiedades etiquetadas) o \emph{Resource Description Framework} (marco
de descripción de recursos, \emph{RDF}).

\emph{GraphDB} adopta la segunda estructura, que consiste en estructurar los
grafos mediante \emph{triples} y \emph{quads}: los \emph{triples} están
compuestos por nodo-arco-nodo y los \emph{quads} complementan a estos
con información de contexto adicional, lo que facilita la división de
los datos en grupos. Esta estrutucta es la ideal para almacenar
ontologías como \emph{AO-CAT}, de ahí que \emph{ARIADNEplus} haya escogido esta
tecnología.

\imagen{triple}{\emph{GraphDB} -- \emph{Triple}.}{0.7}

En la Figura \ref{fig:triple} se ha representado un \emph{triple} que se
correspondería con una parte del grafo asociado a la colección \emph{CIR}
almacenada en este tipo de base de datos. Vemos como se compone de dos
nodos, uno para el sujeto (i.e. \emph{CIR}) y otro para el objeto (i.e. \emph{Scientific
analysis}), unidos por un arco, que sería el predicado
(i.e. \emph{has\_ARIADNE\_subject}).

\section{Conceptos teóricos relativos al desarrollo de la infraestructura}

A continuación se definen aquellos conceptos relacionados con el
desarrollo de la infraestructura.

\subsection{Sistema de gestión de contenidos (\emph{CMS})}

Un sistema de gestión de contenidos o \emph{CMS} \cite{wiki:cms} (\emph{\textbf{C}ontent \textbf{M}anagement \textbf{S}ystem}) es una aplicación \emph{software}, generalmente de tipo \emph{web}, que permite crear un entorno de trabajo para la creación y gestión de contenidos. 

Este tipo de sistemas interactúan con una o varias bases de datos que almacenan el contenido sobre el que se realizan las operaciones de gestión. Además, suelen contar con sistemas que permiten adaptar la aplicación, tanto en diseño como en funcionalidad, de una forma sencilla.

La aplicación escogida para este proyecto (\emph{Omeka Classic} \cite{omeka:web}) se puede catalogar como \emph{CMS}.

\subsection{LAMP}

Las siglas \emph{LAMP} \cite{wiki:lamp} son utilizadas para describir infraestructuras
\emph{software} que hacen uso de cuatro herramientas específicas:

\begin{itemize}
\tightlist
\item
  \emph{\textbf{L}inux} como sistema operativo.
\item
  \emph{\textbf{A}pache} como servidor web.
\item
  \emph{\textbf{M}ysql o \textbf{M}ariaDB} como gestor de base de datos.
\item
  \emph{\textbf{P}HP} como lenguaje de programación.
\end{itemize}

La aplicación \emph{software} escogida requiere dicha infraestructura.

\subsection{Complementos (\emph{Plugins})}

Los complementos, más conocidos como \emph{plugins}, son aplicaciones
que permiten ampliar la funcionalidad básica de un determinado producto
software. Normalmente este tipo de aplicaciones son ejecutadas a través
del \emph{software} principal, interactuando con este a través de una
determinada interfaz.

Mediante este tipo de aplicaciones se han conseguido añadir las funcionalidades
requeridas por el proyecto en la aplicación escogida.

\subsection{\emph{Hooking}}

El término \emph{hooking} \cite{wiki:hook} es empleado para referirse a todas aquellas
técnicas utilizadas para modificar el comportamiento de un sistema
operativo, aplicación u otro componente \emph{software} interceptando
llamadas de función, mensajes o eventos pasados entre componentes
\emph{software}. El código que maneja estos acontecimientos se le
denomina \emph{hook}.

\imagen{hooks}{Ejemplo de \emph{hook}.}{0.7}

Haciendo uso de estos \emph{hooks} se ha conseguido modificar el comportamiento
de la aplicación escogida.

\subsection{Prácticas ágiles}

Durante la fase de desarrollo, se han adoptado una serie de prácticas ágiles que han
contribuído favorablemente al desarrollo del \emph{software}. A
continuación, se explica en qué consiste cada una de ellas.

\subsubsection{Desarrollo iterativo e incremental}

En un desarrollo iterativo e incremental el proyecto se va planificando
en intervalos de tiempo constantes, cada uno de los cuales recibe el
nombre de iteración. En todas las iteraciones se sigue un mismo
procedimiento (de ahí el nombre de iterativo) para conseguir una
funcionalidad determinada del producto que se pretende desarrollar.

En cada iteración, se van completando partes del producto final que son
aptas para ser entregadas al cliente. Este goteo constante de entregas
es el responsable de que a este procedimiento se le denomine
incremental. Para que esto sea posible, se definen unos
objetivos/requisitos al inicio de cada iteración que marcarán la
evolución del proyecto. También se pueden plantear mejoras para
requisitos que se entregaron en iteraciones anteriores.

\subsubsection{Pruebas unitarias}

Las pruebas unitarias permiten comprobar el correcto funcionamiento de
unidades de código fuente. Con el uso de este tipo de pruebas se
pretende asegurar que cada unidad se comporta adecuadamente frente a
distintas situaciones. 

Resulta complicado determinar a qué nos referimos
cuando decimos ``unidad de código'' ya que, por definición, se puede asociar
este concepto tanto a una clase como a un método.

Habitualmente se desarrolla más de una prueba unitaria por unidad de
código. El motivo radica en que una prueba unitaria sólo es capaz de
comprobar el comportamiento de la unidad ante una única entrada. Lo
ideal es comprobar su comportamiento ante todas aquellas entradas que
tengan una probabilidad razonable de hacer que falle. El conjunto de
pruebas que recoge todas estas entradas se le denomina \emph{test
suite}.

\subsubsection{Integración y Despliegue continuo (CI/CD)}

La integración continua (\emph{CI}) es una práctica utilizada en el desarrollo
de \emph{software} mediante la cual es posible automatizar operaciones
tales como la compilación o ejecución de pruebas. Aplicando esta
metodologíam se consigue detectar fallos con mayor rapidez, mejorar la
calidad del código y reducir el tiempo empleado en validar y
publicar nuevas actualizaciones \emph{software}.

El despliegue continuo (CD) se puede considerar como el siguiente paso a
la integración continua, es decir, una vez automatizados los procesos de
compilación y ejecución de pruebas, se procede a automatizar el
despliegue del producto \emph{software} que estemos desarrollando.

\section{Otros conceptos}

En este apartado se recogen todos aquellos conceptos que tienen cierta
relevancia en el proyecto y no han sido expuestos en secciones
anteriores.

\subsection{\emph{Dublin Core}}

\emph{Dublin Core} es un esquema de metadatos elaborado por la
\emph{DCMI} \cite{dc:web}, organización cuya misión
principal es facilitar la compartición de recursos \emph{on-line} por
medio del desarrollo de un modelo de metadatos ``base'', capaz de
proporcionar información descriptiva básica sobre cualquier recurso, sin
importar el formato de origen, área de especialización u origen
cultural. 

Dispone de 15 elementos descriptivos, los cuales pueden ser
repetidos, aparecer en cualquier orden y estar o no presentes
(son opcionales).

\subsection{\emph{Dublin Core Extended}}

Dado que el modelo \emph{Dublin Core} puede resultar algo escueto, se
presenta como solución el esquema \emph{Dublin Core Extended}, el cual
cuenta con los elementos descriptivos del modelo original y, además,
incluye una serie de elementos adicionales/complementarios \cite{dcterms:web}
que satisfacen las necesidades que el modelo original no cubre.

Este modelo ha sido el propuesto para transformar todos los conjuntos de 
datos del \emph{CENIEH} a un único modelo estándar.

\subsection{Interoperabilidad}

La interoperabilidad es la capacidad que tiene un sistema o producto de
compartir datos y posibilitar el intercambio de información y
conocimiento entre ellos \cite{interop:web}.
En lo que respecta a repositorios, se puede conseguir dicha capacidad
haciendo uso de estándares como, por ejemplo, el protocolo
\emph{OAI-PMH}.

\subsection{Protocolo \emph{OAI-PMH}}

El protocolo \emph{Open Archive Initiative-Protocol for Metadata
Harvesting} (\emph{OAI-PMH}) tiene como objetivo desarrollar y promover
estándares de interoperabilidad que faciliten la difusión eficiente de
contenidos en Internet. Permite transmitir metadatos entre diferentes
tipos de infraestructuras \emph{software} (repositorios, gestores, etc.)
siempre y cuando éstos se codifiquen en \emph{Dublin Core}.

Gracias a que la aplicación escogida ofrece este servicio, haciendo uso
del mismo se han podido recolectar todos los metadatos existentes en el
\emph{CIR}. Además, \emph{ARIADNEplus} permite importar metadatos en su catálogo
haciendo uso de este protocolo, por lo que su implantación también abre
otro posible camino de importación.

\imagen{oai-pmh}{Ejemplo básico del protocolo \emph{OAI-PMH}.}{0.5}

\subsection{Geolocalización}

La geolocalización es la capacidad para obtener la ubicación geográfica
real de un objeto \cite{wiki:geo}. Uno de
los requisitos fundamentales del catálodo de \emph{ARIADNEplus} es que todos
los metadatos importados han de estar geolocalizados, es decir, tienen
que tener, al menos, un elemento descriptivo que indique la ubicación
actual del objeto. Nuestra plataforma cuenta con el elemento
\emph{Spatial Coverage} del modelo \emph{Dublin Core Extended} para
cubrir este requisito.

\subsubsection{WSG84}

El \textbf{W}orld \textbf{G}eodetic \textbf{S}ystem \textbf{84} es un
sistema de coordenadas geográficas usado mundialmente para localizar
cualquier punto de la Tierra \cite{wiki:wsg}. Uno de los requisitos del catálogo de
\emph{ARIADNEplus} es que todas aquellas localizaciones señaladas a través de
coordenadas geográficas deben utilizar este sistema.

