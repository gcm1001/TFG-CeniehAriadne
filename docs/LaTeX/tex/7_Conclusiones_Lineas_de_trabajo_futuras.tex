\capitulo{7}{Conclusiones y Líneas de trabajo futuras}

En este último apartado, se exponen las conclusiones extraídas tras un
breve análisis objetivo del trabajo realizado. Además, se proponen
nuevas perspectivas para las posibles líneas de trabajo futuras.

\section{Conclusiones}

A continuación se listan las conclusiones más relevantes que se han
obtenido tras finalizar el proyecto.

\begin{itemize}
\tightlist
\item
  En cuanto a los objetivos generales del proyecto, considero que se han
  cumplido ambos. Los operarios del \emph{CENIEH} cuentan con una
  aplicación que les facilita el proceso de integración de sus datos en
  \emph{ARIADNEplus} y, además, se ha conseguido integrar una de las
  colecciones propuestas.
\item
  Durante la fase de investigación del proyecto, se han aprendido
  multitud de técnicas y conocimientos nuevos relacionados con la
  creación, implementación y gestión de metadatos.
\item
  El ser parte de un proyecto internacional como \emph{ARIADNEplus} nos
  ha permitido conocer nuevos métodos de trabajo como, por ejemplo, la
  utilización de entornos de investigación virtuales (\emph{VREs}).
  Gracias a estos hemos podido comunicarnos con los demás socios del
  proyecto, utilizar servicios y herramientas comunes, y compartir
  recursos digitales de todo tipo.
\item
  En la parte de desarrollo del proyecto se han aplicado la mayoría de
  los conocimientos adquiridos durante el grado. Asímismo, se han
  utilizado otras materias que han requerido un estudio especial como
  \emph{PHP}, \emph{Zend Framework}, \emph{Hooking}, etc.
\item
  El desarrollo de \emph{plugins} o complementos para la adaptación de
  la aplicación propuesta ha supuesto una experiencia totalmente nueva
  que me ha permitido conocer cómo funcionan este tipo de aplicaciones.
\item
  En el proyecto se han aplicado técnicas de integración continua que
  han permitido agilizar muchas de las tareas involucradas en su
  desarrollo, afectando positivamente a la calidad del código y a la
  depuración de errores.
\end{itemize}

\section{Líneas de trabajo futuras}

Se pueden tomar dos caminos distintos para mejorar la aplicación
propuesta:

\begin{enumerate}
\def\labelenumi{\arabic{enumi}.}
\tightlist
\item
  Desarrollar nuevos complementos (\emph{plugins}) que añadan nuevas
  funcionalidades.
\item
  Extender la funcionalidad de los complementos propuestos en este
  proyecto.
\end{enumerate}

A continuación se exponen las funcionalidades que pueden resultar
interesantes añadir en la plataforma.

\begin{itemize}
\tightlist
\item
  Dar algún tipo de soporte a los periodos temporales que pudieran
  aparecer dentro de los (meta)datos. Por ejemplo, representarlos
  gráficamente dentro de una linea temporal.
\item
  Sugerir periodos temporales del cliente \emph{PeriodO} a la hora de
  rellenar el metadato "Temporal Coverage". De esta manera, se podrá
  enriquecer dicho metadato en la fase de integración correspondiente.
\end{itemize}

En cuanto a las posibles mejoras de los complementos:

\begin{itemize}
\item
  Complemento \emph{ARIADNEplus Tracking}: introducir nuevas funciones
  en alguna de las fases de los \emph{tickets}.

  \begin{quote}
  \begin{itemize}
  \tightlist
  \item
    \emph{Fase 1}: Poder editar los ítems desde la misma ventana, sin
    necesidad de desplazarse al gestor de ítems.
  \item
    \emph{Fase 3}: Previsualizar la colección de \emph{periodO} indicada
    por el usuario y poder adjuntar el fichero de definición de mapeo
    desde la misma ventana.
  \item
    \emph{Fase 4}: Previsualizar los ítems publicados en el portal
    fantasma de \emph{ARIADNEplus} a partir del enlace \emph{SPARQL}.
  \end{itemize}
  \end{quote}
\item
  Complemento \emph{Geolocation}: introducir localizaciones con áreas
  poligonales (hasta ahora solo se pueden simples o rectangulares).
\item
  Complemento \emph{Bulk Metadata Editor}: introducir nuevas acciones de
  edición como, por ejemplo, poder asignar más de dos valores a un mismo
  metadato.
\item
  Complemento \emph{OAI-PMH Harvester}: poder programar recolecciones de
  metadatos en determinados intervalos de tiempo.
\item
  Complemento \emph{AutoDublinCore}: \item
  dado que la localización de los datos es imprescindible, crear un
  sistema que en caso de que el metadato que se encarga de ello
  ("\emph{Spatial Coverage}) se encuentre vacío, busque en el contenido
  de los demás metadatos (e.g. \emph{Title}, \emph{Description}, etc.)
  una localización y, en caso de encontrarla, actualizar el contenido
  del metadato con dicha localización.
\item 
  Traducir todos los complementos desarrollados en este proyecto a otros idiomas.
\end{itemize}
