% Options for packages loaded elsewhere
\PassOptionsToPackage{unicode}{hyperref}
\PassOptionsToPackage{hyphens}{url}
%
\documentclass[
]{article}
\usepackage{lmodern}
\usepackage{amssymb,amsmath}
\usepackage{ifxetex,ifluatex}
\ifnum 0\ifxetex 1\fi\ifluatex 1\fi=0 % if pdftex
  \usepackage[T1]{fontenc}
  \usepackage[utf8]{inputenc}
  \usepackage{textcomp} % provide euro and other symbols
\else % if luatex or xetex
  \usepackage{unicode-math}
  \defaultfontfeatures{Scale=MatchLowercase}
  \defaultfontfeatures[\rmfamily]{Ligatures=TeX,Scale=1}
\fi
% Use upquote if available, for straight quotes in verbatim environments
\IfFileExists{upquote.sty}{\usepackage{upquote}}{}
\IfFileExists{microtype.sty}{% use microtype if available
  \usepackage[]{microtype}
  \UseMicrotypeSet[protrusion]{basicmath} % disable protrusion for tt fonts
}{}
\makeatletter
\@ifundefined{KOMAClassName}{% if non-KOMA class
  \IfFileExists{parskip.sty}{%
    \usepackage{parskip}
  }{% else
    \setlength{\parindent}{0pt}
    \setlength{\parskip}{6pt plus 2pt minus 1pt}}
}{% if KOMA class
  \KOMAoptions{parskip=half}}
\makeatother
\usepackage{xcolor}
\IfFileExists{xurl.sty}{\usepackage{xurl}}{} % add URL line breaks if available
\IfFileExists{bookmark.sty}{\usepackage{bookmark}}{\usepackage{hyperref}}
\hypersetup{
  pdftitle={Trabajos relacionados},
  hidelinks,
  pdfcreator={LaTeX via pandoc}}
\urlstyle{same} % disable monospaced font for URLs
\usepackage{longtable,booktabs}
% Correct order of tables after \paragraph or \subparagraph
\usepackage{etoolbox}
\makeatletter
\patchcmd\longtable{\par}{\if@noskipsec\mbox{}\fi\par}{}{}
\makeatother
% Allow footnotes in longtable head/foot
\IfFileExists{footnotehyper.sty}{\usepackage{footnotehyper}}{\usepackage{footnote}}
\makesavenoteenv{longtable}
\setlength{\emergencystretch}{3em} % prevent overfull lines
\providecommand{\tightlist}{%
  \setlength{\itemsep}{0pt}\setlength{\parskip}{0pt}}
\setcounter{secnumdepth}{-\maxdimen} % remove section numbering

\title{Trabajos relacionados}
\author{}
\date{}

\begin{document}
\maketitle

Algunos socios del proyecto ARIADNEplus han adoptado una solución muy
similar a la propuesta en el presente proyecto, es decir, han hecho uso
de aplicaciones \emph{software} de terceros para la gestión de sus
(meta)datos y las han adaptado según sus necesidades. A continuación, se
muestran aquellos casos que guardan una mayor relación con el proyecto.

\hypertarget{casos-similares}{%
\section{Casos similares}\label{casos-similares}}

\hypertarget{fasti-online}{%
\subsection{Fasti Online}\label{fasti-online}}

Fasti Online\footnote{"Fasti Online." \url{http://www.fastionline.org/}}
es un proyecto liderado por la Asociación Internacional de Arqueología
Clásica (AIAC)\footnote{"AIAC -- Associazione Internazionale di
  Archeologia Classica." \url{http://www.aiac.org/}} y el \emph{Center
for the Study of Ancient Italy} (CSAI)\footnote{"CSAI -- Center for the
  Study of Ancient Italy." \url{http://csaitx.org/}} de la Universidad
de Texas, Austin\footnote{"University of Texas at Austin."
  \url{https://www.utexas.edu/}}. Su principal objetivo es proporcionar
una infraestructura \emph{software} que permita almacenar, gestionar y
publicar registros relacionados con la arqueología.

Para tal fin, han utilizado como base la aplicación \emph{software}
denominada \href{https://ark.lparchaeology.com/}{ARK}. Esta es una
aplicación web que provee servicios como la gestión, compartición y
transformación (mapeo) de (meta)datos. Además, la aplicación es de
código abierto, lo que significa que es personalizable y extensible.

La incorporación de \emph{Fasti Online} al proyecto \emph{ARIADNE} y,
posteriormente, al proyecto \emph{ARIADNEplus}, ha impulsado la
implementación de nuevas funcionalidades sobre la aplicación \emph{ARK}
como, por ejemplo, la integración de datos espaciales, nuevos mecanismos
de búsqueda y otros servicios web como, por ejemplo, el protocolo
\emph{OAI-PMH}.

\hypertarget{conicet}{%
\subsection{CONICET}\label{conicet}}

El Consejo Nacional de Investigaciones Científicas y Técnicas
(CONICET)\footnote{"CONICET -- Consejo Nacional de Investigaciones
  Científicas y Técnicas." \url{https://www.conicet.gov.ar/}} es el
principal organismo dedicado a la promoción de la ciencia y la
tecnología en Argentina. Este, al igual que el CENIEH, es una de las
nuevas incorporaciones al proyecto \emph{ARIADNEplus} y, como tal, han
tenido que adaptarse para satisfacer los requisitos del proyecto.

La solución planteada por este organismo es muy similar a la del
presente proyecto. Están desarrollado una infraestructura
\emph{software} que permita a los operarios del CONICET gestionar y
publicar sus conjuntos de datos adoptando un esquema de metadatos
compatible con \emph{ARIADNEPlus}. La aplicación \emph{software} que han
decidido adaptar ha sido \href{https://duraspace.org/dspace/}{Dspace
5.5} . Se puede acceder a su infraestructura desde el siguiente
\href{https://suquia.ffyh.unc.edu.ar/}{enlace} .

\hypertarget{dans}{%
\subsection{DANS}\label{dans}}

DANS (\emph{Data Archiving and Networked Services})\footnote{"DANS --
  Data Archiving and Networked Services." \url{https://dans.knaw.nl/en}}
es una institución de los Paises Bajos cuya misión principal es
proporcionar las herramientas necesarias a investigadores para hacer que
sus datos sean accesibles, interoperables y reutilizables.

La organización DANS es responsable del desarrollo y mantenimiento del
repositorio digital \href{https://dendro.dans.knaw.nl/}{DCCD}, el cual
es considerado como la principal red de (meta)datos
arqueológicos/históricos existente en Europa. Entró en funcionamiento en
2011. Dentro del \emph{DCCD}, laboratorios belgas, daneses, holandeses,
alemanes, letones, polacos y españoles publican contenido fruto de la
investigación de, entre otros: sitios arqueológicos (incluidos paisajes
antiguos), construcciones, pinturas, esculturas e instrumentos
musicales.

Esta organización participó en el proyecto \emph{ARIADNE} y,
actualmente, forma parte del proyecto \emph{ARIADNEplus}. Con el
objetivo de mejorar la integración europea de datos dendrocronológicos
ofrecen, de forma gratuita, la misma solución \emph{software} empleada
en su proyecto \emph{DCCD}, la cual es compatible con el proyecto
\emph{ARIADNE}. Está disponible en
\href{https://github.com/DANS-KNAW/dccd-webui}{Github} .

\hypertarget{comparativa-entre-soluciones-software}{%
\section{\texorpdfstring{Comparativa entre soluciones
\emph{software}}{Comparativa entre soluciones software}}\label{comparativa-entre-soluciones-software}}

\begin{longtable}[]{@{}lllll@{}}
\caption{Comparativa de las principales características de las
aplicaciones \emph{software} escogidas por cada socio.}\tabularnewline
\toprule
Caraterísticas & Omeka Classic (CENIEH) & ARK (Fasti Online) & DSpace
(CONICET) & DCCD (DANS)\tabularnewline
\midrule
\endfirsthead
\toprule
Caraterísticas & Omeka Classic (CENIEH) & ARK (Fasti Online) & DSpace
(CONICET) & DCCD (DANS)\tabularnewline
\midrule
\endhead
Tipo de aplicación & Web & Web & Web & Web\tabularnewline
Lenguaje de programación principal & PHP & PHP & Java &
Java\tabularnewline
Gestión de metadatos & ✔ & ✔ & ✔ & ✔\tabularnewline
Importación masiva de metadatos & ✔ & ✔ & ✔ & ✔\tabularnewline
Exportación masiva de metadatos & ✔ & ✔ & ✔ & ✔\tabularnewline
Edición masiva de metadatos & ✔ & ✘ & ✘ & ✘\tabularnewline
Cobertura espacial & ✔ & ✔ & ✔ & ✔\tabularnewline
Cobertura temporal & ✘ & ✔ & ✘ & ✔\tabularnewline
Protocolo OAI-PMH & ✔ & ✔ & ✔ & ✘\tabularnewline
Herramientas de apoyo en la integración con ARIADNEplus & ✔ & ✘ & ✘ &
✘\tabularnewline
Herramientas para la transformación de metadatos & ✔ & ✔ & ✘ &
✘\tabularnewline
Sistema de usuarios & ✔ & ✔ & ✔ & ✔\tabularnewline
Almacenamiento de ficheros & ✔ & ✔ & ✔ & ✘\tabularnewline
Asistencia técnica gratuita & ✔ & ✘ & ✘ & ✘\tabularnewline
Interfaz pública & ✔ & ✔ & ✔ & ✔\tabularnewline
Interfaz intuitiva & ✔ & ✘ & ✔ & ✘\tabularnewline
Sistema de \emph{plugins} & ✔ & ✘ & ✔ (*) & ✘\tabularnewline
Sistema de plantillas & ✔ & ✘ & ✘ & ✘\tabularnewline
Comunidad de usuarios activa & ✔ & ✘ & ✔ & ✘\tabularnewline
Manuales de documentación detallados & ✔ & ✘ & ✘ & ✘\tabularnewline
Última actualización & 2020 & 2018 & 2020 & 2015\tabularnewline
\bottomrule
\end{longtable}

\emph{(*) Servicio de pago.}

Basándonos en el contenido de la \texttt{compsolsoft}, se listarán los
puntos fuertes y débiles que presenta la aplicación del proyecto frente
a las propuestas de los otros socios.

\hypertarget{puntos-fuertes}{%
\subsection{Puntos fuertes}\label{puntos-fuertes}}

\begin{itemize}
\tightlist
\item
  Gran parte de la configuración de la aplicación puede realizarse desde
  la interfaz gráfica, sin necesidad de modificar ficheros internos que
  requieran un mínimo de conocimiento de la estructura interna de la
  aplicación, como pasa en aplicaciones como \emph{ARK} o \emph{DCCD}.
  Esto facilita en gran medida las labores de configuración de la
  aplicación.
\item
  Al requerir una infraestructura \emph{LAMP} para su despliegue, la
  instalación de la aplicación es relativamente sencilla en comparación
  con las otras aplicaciones. Además, gracias al presente proyecto, es
  posible instalar la aplicación a través de tecnologías como
  \emph{Docker} o \emph{Kubernetes}, facilitando aún más su despliegue.
\item
  De entre todas las soluciones mostradas es, sin duda, la más sencilla
  y segura de adaptar y personalizar. Esto es gracias al sistema de
  complementos (\emph{plugins}) y plantillas (\emph{themes}) que
  incorpora.
\item
  Gracias a las labores de desarrollo llevadas a cabo en el presente
  proyecto, dispone de herramientas de apoyo para la integración de
  conjuntos de datos en ARIADNEplus.
\item
  La comunidad de usuarios con la que cuenta \emph{Omeka Classic} es
  superior a la de sus competidores. Muchos usuarios comparten sus
  propios desarrollos, tanto complementos como plantillas, de forma que
  estos pueden ser reutilizados o incluso mejorados por otros usuarios.
  Además, existe un foro desde donde los expertos de \emph{Omeka},
  incluídos los líderes del proyecto, brindan soporte técnico gratuito a
  otros usuarios de la aplicación.
\item
  La documentación disponible es, tanto para usuarios como para
  desarrolladores, la más clara y detallada de todas las aplicaciones
  mostradas.
\item
  Actualmente el proyecto \emph{Omeka} continúa en desarrollo, es decir,
  siguen saliendo nuevas actualizaciones con mejoras y funcionalidades
  nuevas para la aplicación. Sin embargo, otros proyectos como
  \emph{ARK} o \emph{DCCD} están obsoletos.
\end{itemize}

\hypertarget{puntos-duxe9biles}{%
\subsection{Puntos débiles}\label{puntos-duxe9biles}}

\begin{itemize}
\tightlist
\item
  Actualmente, no dispone de ningún mecanismo que identifique aquellos
  (meta)datos cuyo contenido sea un periodo temporal (e.g. "1190 BCE") y
  los procese de tal forma que estos sean mostrados dentro de una línea
  temporal y a su vez puedan ser un criterio aislado de búsqueda.
\item
  No posee las ventajas que proporciona el lenguaje de programación
  \emph{Java} utilizado tanto en \emph{DSpace} como en \emph{DCCD}. Este
  es más rápido y presenta un mejor rendimiento al ser un lenguaje
  compilado. Además, posee una estructura más ordenada y es mucho más
  seguro que PHP.
\end{itemize}

\end{document}
